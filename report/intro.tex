
\section{Introduction}
In the case of heating devices for fusion experiment, for example, 
 one can find itself dealing with what is called hybrid resonnance. 
 Hybrid resonance may cause Maxwell's equation with strong magnetic 
 fields to develop singular solutions \textcolor{red}{insert refs}. 
 The energy deposit is resonant and may exceed by far the energy 
 exchange which occurs in Landau damping \textcolor{red}{ref}. 
 Contrary to the Landau Dumping, however, 
 hybrid resonnance appears in a simpler model coupling 
 fluid equations with the non electrostatic part of Maxwell equations.
 
 
 As it can be found in \cite{stable_yee_plasma_current}, we consider the non-stationary Maxwell system  
 \begin{align}
-\varepsilon_0 \partial_t \E + \curl\, \Hbf = \J\\
\mu_0 \partial_t \Hbf + \curl\, \E = 0
\end{align}
coupled with a linear current in two dimensions $\J = eN_e \ubf_e$, taking into account the friction between electron and ions
\be
m_e \partial_t \ubf_e =e (\E +\ubf_e \nabla B_0) -m_e \nu \ubf_e. \label{eq:electronmove}
\ee
Here the unknows are the electromagnetc field $(\E,\Hbf)$ with the usual notation $\Hbf = \B/\mu_0$, $\J$ is the electronic current, 
$\ubf_e$ is the velocity of electrons and $B_0$ is the background magnetic field.  
We denote by $e<0$ the value of the charge of electrons and by $\m_e$ the electron mass.
Assuming that $\mathbf{B}_0=\left(0,\; 0,\; B_0\right)$, we obtain the following system of equations 
\begin{align}
\label{eq:main_model}
\begin{split}
\epsilon_0\partial_t E_{x}-\partial_y H=-eN_e u_x,\nonumber\\
\epsilon_0\partial_t E_{y}+\partial_x H=-eN_e u_y,\nonumber\\
\mu_0\partial_t H+\partial_x E_y-\partial_y E_x=0,\\
m_e\partial_t u_x=eE_x+eu_yB_0-\nu m_e u_x,\nonumber\\
m_e\partial_t u_y=eE_y-eu_xB_0-\nu m_e u_y\nonumber
\end{split}
\end{align}
posed in $[-L,\; \mathbb{R}]\times \mathbb{R}$. 
In this work we look at a simplification of this model obtained after 
the Fourier transform in $y$. Denoting by $\theta$ a corresponding Fourier variable, the model (\ref{eq:main_model}) can be rewritten as follows
\begin{align}
\label{eq:main_model_theta}
\begin{split}
\epsilon_0\partial_t E_{x}-\theta H=-eN_e u_x,\nonumber\\
\epsilon_0\partial_t E_{y}+\partial_x H=-eN_e u_y,\nonumber\\
\mu_0\partial_t H+\partial_x E_y-\theta E_x=0,\\
m_e\partial_t u_x=eE_x+eu_yB_0-\nu m_e u_x,\nonumber\\
m_e\partial_t u_y=eE_y-eu_xB_0-\nu m_e u_y.
\end{split}
\end{align}
The energy of this system can be expressed as \cite{}
\begin{align*}
 
\end{align*}
From this it can be seen that when the term $\nu=0$, the electron density $N_e(x)$ vanishes and in plasma there occurs a resonance \cite{Stix}, 
which was recently investigated in \cite{}, in particular, it was shown that the electric field $E_x$ in this case is no longer square 
integrable, and explicit estimates on the behaviour of the solutions of (\ref{eq:main_model_theta}) were given. 

Thus, the goal of this article is two-fold. 
First, following (\ref{}), we investigate the finite element approximation of the 
problem (\ref{eq:main_model_theta}) written in the frequency domain:
\begin{align*}
 
\end{align*}
We prove the well-posedness of this problem for $\nu>0$ in Section \ref{} and 
demonstrate that the use of higher-order (P1) finite elements allows to approximate the singularity of the solution fairly well (Section \ref{}). 
Second, we consider the case $\nu\rightarrow 0$, and study the limiting amplitude solution 
$\lim\limits_{t\rightarrow +\infty}\lim\limits_{\nu\rightarrow 0}\mathbf{E}(t)$ obtained with the help of 
the FDTD discretization (\ref{eq:main_model_theta}), suggested in \cite{}. We compare this result with 
$\hat{\mathbf{E}}\mathrm{e}^{i\omega t}$, computed in the frequency domain, for $\nu\rightarrow 0$, i.e.
$\lim\limits_{\nu\rightarrow 0}\lim\limits_{t\rightarrow+\infty}\mathbf{E}(t)$.

To our knowledge, such numerical studies have not been performed in the existing literature. 





The boundary conditions
\begin{align*}
H|_{-L}=g(t).
\end{align*}
\textcolor{red}{!!!!!!maybe move this next part to later!!!!!!!!}
\subsection{Boundary condition}
Robin Boundary condition at the left boundary of the domain
\be
-curl \E -\imath \lambda\E \wedge \n = g_{inc} = -curl \E_{inc} -\imath \lambda\E_{inc} \wedge \n,
\ee
where $\E_{inc} = \exp \left(\imath \lambda x\right)\begin{pmatrix} E_1\\ E_2 \end{pmatrix}$
$curl \E = 0$ at the right boundary
\begin{remark}
When modelling antennas in plasma physics, in such applications as ITER, a good choice of boundary condition could be the use of $\lambda = \sqrt{alpha(-L)}$
\end{remark}
\textcolor{red}{!!!!!!!!!!!!}