\section{Introduction}
Modelling various phenomena in plasmas is of practical importance for developing new sources of energy 
based on plasma fusion, see the ITER project\footnote{www.iter.org}. 
In particular, this article concentrates on studying a phenomenon of hybrid resonance \cite{Stix}, 
which is observed in experiments (see \cite{reflectometers_2006, reflectometers_2010, Dumont_2005}) and 
mathematically is described as the non-regularity of
the solutions of Maxwell equations in plasmas under strong background magnetic field \cite{Despres_2014}. 
The energy deposit is resonant and may exceed by far the energy 
exchange which occurs in Landau damping, see \cite{Freidberg_2007,Mouhot_2011}. 
Contrary to the Landau damping, however, 
hybrid resonance appears in a simpler model coupling 
fluid equations with the non electrostatic part of Maxwell equations.


We consider the model of cold plasma of \cite{Stix} which is described by the non-stationary Maxwell system  
\begin{align}
-\varepsilon_0 \partial_t \E + \curl\, \Hbf = \J\\
\mu_0 \partial_t \Hbf + \curl\, \E = 0
\end{align}
coupled with a linear current in two dimensions $\J = eN_e \ubf_e$, and takes into account the friction  $\nu$ between particles
\be
m_e \partial_t \ubf_e =e (\E +\ubf_e \nabla B_0) -m_e \nu \ubf_e. \label{eq:electronmove}
\ee
The unknows are the electromagnetic field $(\E,\Hbf)$ with the usual notation $\Hbf = \B/\mu_0$, 
the electronic current $\J$, 
and the velocity of electrons $\ubf_e$. Here $B_0$ is the background magnetic field, typically assumed to be uniform in time and space.  
We denote by $e<0$ the value of the charge of electrons, $m_e$ the electron mass and $N_e$ the electron density that in general 
depends on spatial variables and is uniform in time. 
Without loss of generality, we set $\mathbf{B}_0=\left(0,\; 0,\; B_0\right)$, which allows to obtain the following system of equations 
\begin{align}
\label{eq:main_model}
\begin{split}
\epsilon_0\partial_t E_{x}-\partial_y H=-eN_e u_x,\\
\epsilon_0\partial_t E_{y}+\partial_x H=-eN_e u_y,\\
\mu_0\partial_t H+\partial_x E_y-\partial_y E_x=0,\\
m_e\partial_t u_x=eE_x+eu_yB_0-\nu m_e u_x,\\
m_e\partial_t u_y=eE_y-eu_xB_0-\nu m_e u_y
\end{split}
\end{align}
posed in $[-L,\; \mathbb{R}]\times \mathbb{R}$. 
The energy of this system for $\nu=0$ in a domain $\Omega$ can be expressed as \cite{stable_yee_plasma_current}
\begin{align*}
{\mathcal E}(t)= \int_\Omega \left(
\frac{\epsilon_0 |\E(t,\x)|^2}{2}+\frac{ |\B(t,\x)|^2}{2\mu_0}+\frac{m_e|\J (t,\x)|}{2|e|N_e(\x)} \mathrm{d}\x.
\right)
\end{align*}
From this it can be seen that if the electron density $N_e(x)$ vanishes, 
the energy blows up~\cite{stable_yee_plasma_current}. 
Mathematically, this effect was recently investigated in \cite{Despres_2014}, 
in particular, it was shown that the electric field $E_x$ in this case is no longer square 
integrable, and explicit estimates on the behaviour of the solutions of \eqref{eq:main_model} in 1D were given. 

In this work we look at a simplification of this model obtained after 
the Fourier transform of (\ref{eq:main_model}) in $y$. 



The goal of this article is two-fold. 
First, we investigate the finite element approximation of the 1D 
problem \eqref{eq:main_model} written in the frequency domain ($\partial_t\rightarrow -i\omega$).
\mrev{More precisely, 
\begin{align*}
%\label{eq:main_frequency_domain_1}
\operatorname{curl}\operatorname{curl}\hat{\mathbf{E}}-
\epsilon(\omega)\hat{\mathbf{E}}=0,\\
\epsilon(\omega)=\left(
\begin{matrix}
 \tilde{\alpha} & i\tilde{\delta} \\
 -i\tilde{\delta}  & \tilde{\alpha}
\end{matrix}
\right),\\
\tilde{\alpha}(\omega)=\frac{\omega^2}{c^2}\left(1-\frac{\tilde{\omega}\omega_p^2}{\omega(\tilde{\omega}^2-\omega_c^2)}\right),\qquad 
\tilde{\delta}(\omega)=\frac{\omega_c\omega_p^2}{\omega(\tilde{\omega}^2-\omega_c^2)},\\
\tilde{\omega}=\omega+i\nu,\qquad \omega_c=\mrev{\frac{eB_0}{m_e}},\qquad \omega_p^2=\frac{e^2 N_e}{\epsilon_0 m_e}.
\end{align*}
After expansion in $\nu\rightarrow 0$ and keeping lower-order terms only
\begin{align*}
 \tilde{\alpha}(\omega)=\alpha(\omega)+i\nu\frac{\omega_p^2(\omega^2+\omega_c^2)}{(\omega^2-\omega_c^2)^2}\frac{\omega^2}{c^2},\qquad \alpha(\omega)=\frac{\omega^2}{c^2}\left(1-\frac{\omega_p^2}{\omega^2-\omega_c^2}\right),\\
 \tilde{\delta}(\omega)=\delta(\omega)-\frac{2i\omega^3\nu}{c^2(\omega^2-\omega_c^2)^2},\qquad \delta(\omega)=\frac{\omega\omega_c\omega_p^2}{c^2(\omega^2-\omega_c^2)}.
\end{align*}
Then the limiting absorption principle can be obtained leaving only the diagonal terms in the above expansion (see \cite{Despres_2014}):
\begin{align}
\label{eq:main_frequency_domain}
\operatorname{curl}\operatorname{curl}\hat{\mathbf{E}}-\frac{\omega^2}{c^2}
\left(\epsilon_0(\omega)+\nu Id\right)\hat{\mathbf{E}}=0,\\
\nonumber
\epsilon_0(\omega)=\left(
\begin{matrix}
 \alpha & i\delta \\
 -i\delta & \alpha 
\end{matrix}
\right)
\end{align}
}
\textcolor{blue}{Along with this frequency domain study, we investigate the time dependant problem.}\urev{ We assume $\alpha(\omega)$ and $\delta(\omega)$ be sufficiently smooth, i.e. bounded and continuous in  
$\left[-L,\; \mathbb{R}\right)$.}
In particular, we again investigate this problem in one dimension, performing the Fourier transform in $y$.
We prove the well-posedness of this problem for $\nu>0$ in Section~\ref{sec:wellposedness} and 
demonstrate that the use of higher-order (P1) finite elements allows to approximate the singularity 
of the solution fairly well (Section \ref{sec:freq_dep}). 
Second, we consider the case $\nu\rightarrow 0$, and study the limiting amplitude solution 
$\lim\limits_{t\rightarrow +\infty}\lim\limits_{\nu\rightarrow 0}\mathbf{E}(t)$ obtained with the help of 
the FDTD discretization of \eqref{eq:main_model}, suggested in \cite{stable_yee_plasma_current}. 
We compare this result with 
$\hat{\mathbf{E}}\mathrm{e}^{i\omega t}$, computed in the frequency domain, for $\nu\rightarrow 0$, i.e.
$\lim\limits_{\nu\rightarrow 0}\lim\limits_{t\rightarrow+\infty}\mathbf{E}(t)$.

To our knowledge, such numerical studies have not been performed in the existing literature. 

To solve the problem (\ref{eq:main_model}) numerically, we equip it with the following boundary conditions.
At the left boundary of the domain we choose Robin boundary conditions 
\be
-curl \E -\imath \lambda\E \wedge \n = g_{inc} = -curl \E_{inc} -\imath\lambda\E_{inc} \wedge \n,
\ee
where $\E_{inc} = \exp \left(\imath \lambda x\right)\begin{pmatrix} E_1\\ E_2 \end{pmatrix}$. We truncate the domain 
$(-L,\; \mathbb{R})$ to $(-L,\; H)$ and set on the right boundary $\operatorname{curl} \E = 0$. 
\begin{remark}
	When modelling antennas in plasma physics, in such applications as ITER, 
	a good choice of boundary condition could be $\lambda = \sqrt{\alpha(-L)}$
\end{remark}
