\section{Introduction}

We consider the set of Maxwell equations with a linear current in two dimensions \cite{stable_yee_plasma_current}
 with absorption terms
 \begin{align*}
\epsilon_0\partial_t E_{x}-\partial_y H=-eN_e u_x,\\
\epsilon_0\partial_t E_{y}+\partial_x H=-eN_e u_y,\\
\mu_0\partial_t H+\partial_x E_y-\partial_y E_x=0,\\
m_e\partial_t u_x=eE_x+eu_yB_0-\nu m_e u_x,\\
m_e\partial_t u_y=eE_y-eu_xB_0-\nu m_e u_y
\end{align*}
posed in $[-L,\; \mathbb{R}]\times \mathbb{R}$. The boundary conditions 
\begin{align*}
 H|_{-L}=g(t).
\end{align*}
\subsection{Boundary condition}
Robin Boundary condition at the left boundary of the domain
\be
-curl \E -\imath \lambda\E \wedge \n = g_{inc} = -curl \E_{inc} -\imath \lambda\E_{inc} \wedge \n,
\ee
where  $\E_{inc} = \exp \left(\imath \lambda x\right)\begin{pmatrix} E_1\\ E_2 \end{pmatrix}$

$curl \E = 0$ at the right boundary
\begin{remark}
When modelling antennas in plasma physics, in such applications as ITER, a good choice of boundary condition could be the use of $\lambda = \sqrt{alpha(-L)}$
\end{remark}