\section{Introduction}



 In the case of heating devices for fusion experiment, for example, one can find itself dealing with what is called hybrid resonnance. Hybrid resonnance may cause Maxwell's equation with strong magnetic fields to develop singular solutions \textcolor{red}{insert refs}. The energy deposit is resonant and may exceed by far the energy exchange which occurs in Landau damping \textcolor{red}{ref}. Contrary to the Landau Dumping, however, hybrid resonnance appears in a simpler model coupling fluid equations with the non electrostatic part of Maxwell equations.
 As it can be found in \cite{stable_yee_plasma_current}, we consider the non-stationary Maxwell system 
 \begin{align}
 -\varepsilon_0 \partial_t \E + \curl\, \Hbf = \J\\
 \mu_0 \partial_t \Hbf + \curl\,  \E = 0
 \end{align}
 coupled with a linear current in two dimensions $\J = eN_e \ubf_e$, taking into account the friction between electron and ions
 
 \be
 m_e \partial_t \ubf_e =e (\E +\ubf_e \nabla B_0) -m_e \nu \ubf_e. \label{eq:electronmove}
 \ee
 Here the unknows are the electromagnetc field $(\E,\Hbf)$ with the usual notation $\Hbf = \B/\mu_0$, $\J$ is the electronic current. We denote by $e<0$ the value of the charge of electrons and by $\m_e$ the electron mass0
  \begin{align}
\epsilon_0\partial_t E_{x}-\partial_y H=-eN_e u_x,\nonumber\\
\epsilon_0\partial_t E_{y}+\partial_x H=-eN_e u_y,\nonumber\\
\mu_0\partial_t H+\partial_x E_y-\partial_y E_x=0,\\
m_e\partial_t u_x=eE_x+eu_yB_0-\nu m_e u_x,\nonumber\\
m_e\partial_t u_y=eE_y-eu_xB_0-\nu m_e u_y\nonumber
\end{align}
posed in $[-L,\; \mathbb{R}]\times \mathbb{R}$. The boundary conditions 
\begin{align*}
 H|_{-L}=g(t).
\end{align*}

\textcolor{red}{!!!!!!maybe move this next part to later!!!!!!!!}
\subsection{Boundary condition}
Robin Boundary condition at the left boundary of the domain
\be
-curl \E -\imath \lambda\E \wedge \n = g_{inc} = -curl \E_{inc} -\imath \lambda\E_{inc} \wedge \n,
\ee
where  $\E_{inc} = \exp \left(\imath \lambda x\right)\begin{pmatrix} E_1\\ E_2 \end{pmatrix}$

$curl \E = 0$ at the right boundary
\begin{remark}
When modelling antennas in plasma physics, in such applications as ITER, a good choice of boundary condition could be the use of $\lambda = \sqrt{alpha(-L)}$
\end{remark}
\textcolor{red}{!!!!!!!!!!!!}