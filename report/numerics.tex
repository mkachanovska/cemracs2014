\section{Discretization}
Clearly, the solution of the problem 
\begin{align*}
a(u, v)=(f,v)\text{ for all } v=(v_{1},v_{2})^{T}\in \mathbf{V}=V_1\times V_2, 
\end{align*}
is equivalent to the solution of
\begin{align*}
a(u,(v_{1},0)^{T})=(f,(v_{1},0)^{T}),\text{ for all } v_1\in V_1,\\
a(u,(0,v_{2})^{T})=(f,(0,v_{2})^{T}),\text{ for all } v_2\in V_2.
\end{align*}
Thus, the variational formulation (\ref{}) can be rewritten as a system of two equations:
\begin{align*}
\begin{split}
i\theta \displaystyle \int_{-L}^H (E_2' -\imath\theta E_1)\overline{\tilde E_1} - \int_{-L}^H \left((\epsilon_0 +\imath\nu \operatorname{Id}) \E\right)_{1} \overline{\tilde E}_{1}
&=0,\\
\displaystyle \int_{-L}^H (E_2' -\imath\theta E_1)\tilde {E_2'} - \int_{-L}^H\left( (\epsilon_0 +\imath\nu\operatorname{Id}) \E\right)_{2}\overline{\tilde{E}}_{2}
  - \imath \sqrt{\alpha(-L)} E_2 (-L) \tilde E_2 (-L) &= -g_{inc} (-L) \overline{( \tilde E_2(-L) )},\\
  \text{ for all } \tilde E_{1}\in L_{2}(\Omega), \; \tilde{E}_{2}\in H_{1}(\Omega).
  \end{split}
\end{align*}
Inserting an explisit expression for the dielectric permittivity tensor 
\be \label{eq:epsilonmu}
 \epsilon_0(x) =
\left(
\begin{array}{cc}
\alpha(\x) & \imath \delta (\x)\\
-\imath \delta (\x) & \alpha(\x)
\end{array}
\right)
\ee
into the above, we obtain
\begin{align}
\label{eq:var_form2}
\begin{split}
 i\theta \displaystyle \int_{-L}^H (E_2' -\imath\theta E_1)\overline{\tilde E_1} - 
 \int_{-L}^H \left((\alpha+i\nu)E_1+\imath\delta E_2\right)_{1} \overline{\tilde E}_{1}
&=0,\\
\int_{-L}^H (E_2' -\imath\theta E_1)\tilde {E_2'} -
\int_{-L}^H\left( -\imath \delta E_1+(\alpha+i\nu) E_2\right)_{2}\overline{\tilde{E}}_{2}
  - \imath \sqrt{\alpha(-L)} E_2 (-L) \tilde E_2 (-L) &= -g_{inc} (-L) \overline{( \tilde E_2(-L) )},\\
  \text{ for all } \tilde E_{1}\in L_{2}(\Omega), \; \tilde{E}_{2}\in H_{1}(\Omega).
  \end{split}
\end{align}


\begin{remark}
Let us assume that the operator
\begin{align*}
M\psi = \frac{i\delta}{(\alpha+i\nu)}\psi
\end{align*}
defines an isomorphism from $L_{2}(\Omega)$ to $L_{2}(\Omega)$. 
Given $E_1,\; E_2\in L_{2}(\Omega)$, the function $\mathcal{F}=\frac{i\delta}{(\alpha+\imath\nu)}\left((\alpha+\imath\nu)E_{1}+i\delta E_{2}\right)$ also belongs to $L_{2}(\Omega)$. 
If $E_1,\; E_2$ satisfy the problem (\ref{eq:var_form2}) for $\theta=0$, the function $\mathcal{F}=0$, 
i.e.
\begin{align*}
 (\mathcal{F},\tilde{E}_{2})_{L_{2}}=\int_{-L}^{H}\left(\imath\delta E_{1}-\delta^2(\alpha+i\nu)^{-1} E_{2}\right)\overline{\tilde E}_{2}=0,
\end{align*}
for all $\overline{\tilde E}_2\in H_{1}(\Omega)$. Adding this to the second expression of (\ref{eq:var_form2}), we obtain 
the variational formulation for $E_2$, which coincides with the variational formulation for the Helmholtz equation. The well-posedness 
of this problem had been studied in \cite{}. 
Hence, we can add the second expression (\ref{eq:E2}) to
\begin{align*}
\int_{-L}^{H}\left(i\delta E_{1}-\delta^2(\alpha+i\nu)^{-1} E_{2}\right)\overline{\tilde E}_{2}=0
\end{align*}
to obtain the variational formulation for $E_{2}$, which will coincide with the variational formulation for the Helmholtz equation, and thus will be well-posed 
(see the thesis \cite{thesis_lmig}).
The estimates from \cite{thesis_lmig} can be employed to bound $\|E_{1}\|_{L_{2}}$.
\end{remark}


Introducing two basis spaces, $V_{E_{1}}=\{\psi_{j}\}_{j=1}^{N_{1}}$ and $V_{E_{2}}=\{\phi_{i}\}_{i=1}^{N_{2}}$, we look for the solution of the problem (\ref{}) in the form:
\begin{align*}
E_{1}=\sum\limits_{k=1}^{N_{1}}e_{1k}\psi_{k},\; E_{2}=\sum\limits_{k=1}^{N_{2}}e_{2k}\phi_{k}.
\end{align*}
Substituting $\tilde{E}_{1}=\psi_{m}, \; m=1,\ldots,N_{1}$ and $\tilde{E}_{2}=0$ into the variational formulation (\ref{}), we obtain:
\begin{align*}
i\theta\sum\limits_{k=1}^{N_{2}}e_{2k}\int\limits_{-L}^{H}\phi'_{k}\bar{\psi}_{m}dx+\theta^2\sum\limits_{k=1}^{N_{1}}e_{1k}\int\limits_{-L}^{H}\psi_{k}\bar{\psi}_{m}dx\\
-\sum\limits_{k=1}^{N_{1}}e_{1k}\int\limits_{-L}^{H}(\alpha(x)+i\nu)\psi_{k}\bar{\psi}_{m}dx-i\sum\limits_{1}^{N_{2}}e_{2k}\int\limits_{-L}^{H}\delta(x)\phi_{k}(x)\bar{\psi}_{m}dx=0.
\end{align*}
Similarly, for $\tilde{E}_{1}=0$ and $\tilde{E}_{2}=\phi_{\ell},\; \ell=1,\ldots, N_{2}$:
\begin{align*}
\sum\limits_{k=1}^{N_{2}}e_{2k}\int\limits_{-L}^{H}\phi'_{k}(x)\bar{\phi}'_{\ell}(x)dx-i\theta\sum\limits_{k=1}^{N_{1}}e_{1k}\int\limits_{-L}^{H}\psi_{k}(x)\bar{\phi}'_{\ell}(x)dx\\
+i
\sum\limits_{k=1}^{N_{1}}e_{1k}\int\limits_{-L}^{H}\delta(x)\psi_{k}\bar{\phi}_{m}dx-\sum\limits_{k=1}^{N_{2}}e_{2k}\int\limits_{-L}^{H}(\alpha(x)+i\nu)\phi_{k}\bar{\phi}_{m}dx\\
-
i\lambda\sum\limits_{k=1}^{N_{2}}e_{2k}\phi_{k}(-L)\bar{\phi}_{m}(-L)=-g_{inc}(x)\bar{\phi}_{m}(-L).
\end{align*}
After introduction 
\begin{align*}
\left(K^{\psi,\phi'}\right)_{mk}=\int\limits_{-L}^{H}\bar{\psi}_{m}\phi'_{k}dx,\qquad \left(M^{\psi}\right)_{mk}=\int\limits_{-L}^{H}\psi_{k}\bar{\psi}_{m}dx,\\
\left(M^{\alpha,\psi}\right)_{mk}=\int\limits_{-L}^{H}(\alpha(x)+i\nu)\psi_{m}\bar{\psi}_{k}dx, \qquad \left(M^{\delta,\psi,\phi}\right)_{mk}=\int\limits_{-L}^{H}\delta(x)\bar{\psi}_{m}\phi_{k}dx,\\
K_{\ell k}=\int\limits_{-L}^{H}\phi'_{k}(x)\bar{\phi}'_{\ell}(x)dx,\qquad \left(M^{\alpha,\phi}\right)_{\ell k}=\int\limits_{-L}^{H}(\alpha(x)+i\nu)\bar{\phi}_{\ell}\phi_{k}dx,\\
I_{km}^{\Gamma}=\bar{\phi}_{m}(-L)\phi_{k}(-L),\\
\boldsymbol{e}_{1}=\left(e_{11},\ldots,e_{1 N_{1}}\right)^{T},\; \boldsymbol{e}_{2}=\left(e_{21},\ldots,e_{2 N_{1}}\right)^{T},\\
\boldsymbol{0}_{n} \text{ is an $n$-dimensional zero column vector}.
\end{align*}
the above system of equations can be rewritten in an antisymmetric block form:
\begin{align*}
\left(\begin{matrix}
\theta^2 M_{\psi}-M^{\alpha,\psi} & i\theta K^{\psi,\phi'}-i M^{\delta,\psi,\phi} \\
-i\theta (K^{\psi,\phi'})^{*}+i (M^{\delta,\psi,\phi})^{*} & K-M^{\alpha,\phi}-i\lambda I^{\Gamma}
\end{matrix}\right)
\left(
\begin{matrix}
\boldsymbol{e}_1\\
\boldsymbol{e}_2
\end{matrix}
\right)=-g_{inc}(-L)
\left(
\begin{matrix}
\boldsymbol{0}_{N_{1}}\\
\bar{\phi}_{1}(-L)\\
\bar{\phi}_{2}(-L)\\
\vdots\\
\bar{\phi}_{N_{2}}(-L)
\end{matrix}
\right).
\end{align*}
This expression greatly simplifies when choosing $V_{E_{1}}=V_{E_{2}}=\left(\phi_{m}\right)_{m=1}^{N_{2}}$ and in the case $\theta=0$:
\begin{align}
\label{eq:simple_system}
\left(\begin{matrix}
M^{\alpha,\phi} & i M^{\delta,\phi,\phi} \\
i (M^{\delta,\phi,\phi})^{*} & K-M^{\alpha,\phi}-i\lambda I^{\Gamma}
\end{matrix}\right)
\left(
\begin{matrix}
\boldsymbol{e}_1\\ 
\boldsymbol{e}_2
\end{matrix}
\right)=-g_{inc}(-L)
\left(
\begin{matrix}
\boldsymbol{0}_{N_{1}}\\
\bar{\phi}_{1}(-L)\\
\bar{\phi}_{2}(-L)\\
\vdots\\
\bar{\phi}_{N_{2}}(-L)
\end{matrix}
\right).
\end{align}
In all numerical experiments, we make use of the formulation, where the Lagrange piecewise-linear ('hat') finite elements are used as a 
basis and trial space (see \cite{brenner}). We apply permutation to the above system 
to obtain a 7-diagonal Hermitian matrix and solve the system with the Gauss back substitution algorithm. 




\section{Numerical Experiments}
This section is organized as follows. The first part is dedicated to the numerical implementation
of the frequency domain formulation (\ref{}). We study the convergence of this formulation and the behaviour 
of the numerical solution as the absorption parameter $\nu$ tends to zero. The experiments in this section were 
performed on a laptop with 2.6GHz Intel Core i5 CPU, with the help of the code written in Octave (compatible with Matlab). 

The second part of the section deals with the question of the equivalence of the limiting absorption and limited amplitude 
principle. We compare the solutions obtained as $\nu\rightarrow 0$ with the help of our frequency domain code 
and of the time domain code (computed for large values of time). 
The time-domain code implements the scheme described in Section \ref{} and 
is written in Fortran. 
\subsection{Frequency Domain Problem}
\subsubsection{Validity of Implementation}
\subsubsection{Solution of $X$-Mode Prolbem}

Let us consider the case of the resonance, more precisely, we consider sufficiently smooth
$\alpha,\delta$, s.t. $\alpha(0)=0$ and $\delta(0)\neq 0$, and the solvability conditions 
of Theorem \ref{}  are satisfied. 
For simplicity, let us consider 
\begin{align}
\label{eq:cond}
 \alpha(x)=-x \text{  in some neighbourhood of $0$ }.
\end{align}


Given the space $\mathbf{V}_{h}=S_{h}^{1}\times S_{h}^{1}$, with $S_{h}^{1}$ consisting of piecewise-linear (hat) functions, we look for a ratio $h(\nu)$ that would ensure that the absolute error 
\begin{align}
\label{eq:problem1}
\|E^{\nu}_{1}-E^{\nu,h}_{1}\|_{L_{2}(\Omega)}<\epsilon,
\end{align}
given a fixed value of $\epsilon>0$. 

We use the following well-known facts:
\begin{itemize}
 \item The C\'ea's lemma applied to the problem (\ref{}); here $C_c$ is the continuity and $C_i$ is the coercivity constants:
\begin{align}
\label{eq:cea}
\begin{split}
 \|\mathbf{E}^{\nu}-\mathbf{E}^{h,\nu}\|_{V}\leq \frac{C_c}{C_i}\min_{\mathbf{v}\in V_h}\|\mathbf{E}-\mathbf{v}\|_{V}
 \leq \nu^{-1}\min_{\mathbf{v}\in V_h}\|\mathbf{E}-\mathbf{v}\|_{V}.
 \end{split}
\end{align}
The last inequality follows from Theorem \ref{}.  
\item The form of the exact solution to the problem (\ref{}):
\begin{align}
\label{eq:exact}
 E_{1}^{\nu}=E_{2}^{\nu}\frac{\delta}{\alpha+i\nu}=\frac{f(x)}{\alpha(x)+i\nu},
\end{align}
for some $f(x)\in L_{2}(\Omega)$.
\item The estimate from \cite[Chapter 0]{Brenner_Scott} on the rate of convergence of the interpolation 
\begin{align}
\label{eq:bsc}
 \|v-I^{h}v\|_{L_{2}(\Omega)}\leq Ch^2|v''|_{L_{2}(\Omega)},\; C>0,\\
 %\|v-I^{h}v\|_{H_{1}(\Omega)}\leq Ch|v''|_{L_{2}(\Omega)},\; C>0, 
\end{align}
where $I^{h}v$ is an interpolation operator onto $S_{h}^{1}$.
\end{itemize}
For $E_{1}^{\nu},\;f(x)$ in (\ref{eq:exact}) being sufficiently smooth, 
\begin{align*}
 \frac{d^2}{dx^2}E_{1}^{\nu}=\frac{f''}{\alpha+i\nu}-2\frac{f'\alpha'}{(\alpha+i\nu)^2}+\frac{f\alpha''}{(\alpha+i\nu)^3},
\end{align*}
from which, together with (\ref{eq:cond}), it follows that there exists $c>0$ s.t. for all sufficiently small $\nu$ 
\begin{align*}
 \left|  \frac{d^2}{dx^2}E_{1}^{\nu}\right|_{L_2}^{2}&\leq c\int\limits_{\Omega}\frac{1}{(x^2+\nu^2)^{3}}dx
 %\\
 %&=c\left.
 %\left(\frac{x}{4(x^2+\nu^2)^2}\nu^{-2}+\frac{3x}{8(x^2+\nu^2)}\nu^{-4}+\frac{3}{8}\nu^{-5}\operatorname{atan}\frac{x}{\nu}\right)\right|_{-L}^{H}
 \leq C\nu^{-5},\; 
\end{align*}
where $C>0$ does not depend on $\nu$. This, together with the estimates (\ref{eq:cea}) and (\ref{eq:bsc}), results in 
\begin{align*}
 \|E^{\nu}_{1}-E^{\nu,h}_{1}\|_{L_{2}(\Omega)}\leq C\nu^{-\frac{7}{2}}h^2,
\end{align*}
from which it follows that to ensure (\ref{eq:problem1}) $h$ should be chosen as $\alpha_{\epsilon}\nu^{\frac{7}{4}}$, 
where $\alpha_{\epsilon}>0$ depends on $\epsilon$ but does not depend on $\nu$. 

Let us check whether this holds true. 
To do so, we conduct the following numerical experiment.



We can show that 
\begin{align*}
 \|E^{\nu}_{1}\|_{L_{2}(\Omega)}\leq \frac{C}{\sqrt{\nu}},\; C>0, 
\end{align*}
and thus the relative error control
\begin{align*}
 \frac{\|E^{\nu}_{1}-E^{\nu,h}_{1}\|_{L_{2}(\Omega)}}{\|E^{\nu}_{1}\|_{L_{2}(\Omega)}}\leq \epsilon
\end{align*}
can be ensured by choosing $h$ as $\beta_{\epsilon}\nu^{\frac{3}{4}}$.

\subsection{Discretization of the Time Domain Problem}
\section{Numerical Experiments}
\subsection{Frequency Domain}
\subsection{Time Domain}