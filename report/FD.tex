\section{Frequency domain study}

\mrev{
Consider the following problem: 
\begin{align*}
%\label{eq:main_frequency_domain_1}
\operatorname{curl}\operatorname{curl}\hat{\mathbf{E}}-
\epsilon(\omega)\hat{\mathbf{E}}=0,\\
\epsilon(\omega)=\left(
\begin{matrix}
 \tilde{\alpha} & i\tilde{\delta} \\
 -i\tilde{\delta}  & \tilde{\alpha}
\end{matrix}
\right),\\
\tilde{\alpha}(\omega)=\frac{\omega^2}{c^2}\left(1-\frac{\tilde{\omega}\omega_p^2}{\omega(\tilde{\omega}^2-\omega_c^2)}\right),\qquad 
\tilde{\delta}(\omega)=\frac{\omega_c\omega_p^2}{\omega(\tilde{\omega}^2-\omega_c^2)},\\
\tilde{\omega}=\omega+i\nu,\qquad \omega_c=\mrev{\frac{eB_0}{m_e}},\qquad \omega_p^2=\frac{e^2 N_e}{\epsilon_0 m_e}.
\end{align*}
After series expansion as $\nu\rightarrow 0$ gives at leading orders
\begin{align*}
 \tilde{\alpha}(\omega)=\alpha(\omega)+i\nu\frac{\omega_p^2(\omega^2+\omega_c^2)}{(\omega^2-\omega_c^2)^2}\frac{\omega^2}{c^2},\qquad \alpha(\omega)=\frac{\omega^2}{c^2}\left(1-\frac{\omega_p^2}{\omega^2-\omega_c^2}\right),\\
 \tilde{\delta}(\omega)=\delta(\omega)-\frac{2i\omega^3\nu}{c^2(\omega^2-\omega_c^2)^2},\qquad \delta(\omega)=\frac{\omega\omega_c\omega_p^2}{c^2(\omega^2-\omega_c^2)}.
\end{align*}
Then the limiting absorption principle can be obtained leaving only the diagonal terms \textcolor{green}{??NOT CLEAR} in the above expansion (see \cite{Despres_2014}):
\begin{align}
\label{eq:main_frequency_domain}
\operatorname{curl}\operatorname{curl}\hat{\mathbf{E}}-\frac{\omega^2}{c^2}
\left(\epsilon_0(\omega)+\nu Id\right)\hat{\mathbf{E}}=0,\\
\nonumber
\epsilon_0(\omega)=\left(
\begin{matrix}
 \alpha & i\delta \\
 -i\delta & \alpha 
\end{matrix}
\right)
\end{align}
}
\urev{ We assume $\alpha(\omega)$ and $\delta(\omega)$ be sufficiently smooth, i.e. bounded and continuous in  
$\left[-L,\; \mathbb{R}\right)$.}



%%%%%%%%%%%%%%%%%%%%%% 
\subsection{1D case}
%%%%%%%%%%%%%%%%%%%%%% 
\label{sec:wellposedness}

In order to reduce problem \eqref{eq:main_frequency_domain} to one dimension, perform a Fourier transform in the $y$ variable, denoting by $\theta$ the corresponding Fourier variable.
The resulting one dimensional system has the following variational formulation:
\begin{align}
\begin{array}{l}
\displaystyle \int_{-L}^H (E_y' -\imath\theta E_x)\overline{(\tilde E_y' -\imath \theta \tilde E_x)} - \int_{-L}^H (\eps_0 +\imath\nu Id) \E \cdot \overline{\tilde \E}
\\ \displaystyle  - \imath \lambda E_y (-L)\overline{ \tilde E_y (-L)} = -g_{inc} (-L) \overline{( \tilde E_y(-L) )} \label{vf1dcase}.
\end{array}
\end{align}
\urev{In this and further sections, where it is not ambiguous, we will use the notation $E_{x,y}$ instead of $\hat{E}_{x,y}$ for convenience.}
For the sake of clarity, consider the following notation: 
\begin{df}
The bilinear form $a$ and the linear form $l$ are such that
\be
a(\ubf,\vbf) = a_1 (\ubf,\vbf) +\imath a_2(\ubf,\vbf)\  \text{ and } \  l(\vbf) = -g_{inc} (-L) \overline{(v_2(-L) )} 
\ee
where $a_1= a_1^*$ and $a_2=a_2^*$ are hermitian. 

Let us denote $\Omega=(-L,\; H)\in\mathbb R$ and introduce the function space $\mathbf{V}=L_{2}(\Omega)\times H_{1}(\Omega)$, 
equipped with the norm
\begin{align*}
 \|\mathbf{v}\|_{\mathbf{V}}^2= \|v_1\|_{L_{2}(\Omega)}^2+\|v_2\|_{H^{1}(\Omega)}^2.
\end{align*}
Moreover the spectral radius of $\eps_0$ is denoted $\heps =  \|\rho(\eps_0)\|_{L^\infty}$.
\end{df}

We now prove the following result.
\begin{lemma}
\label{lemma:well_posedness}
If $\heps $  is bounded, the bilinear form 
\begin{align*}
 a(\ubf,\vbf):  \mathbf{V}\times \mathbf{V}\rightarrow \mathbb{C}
\end{align*}
is continuous and coercive for all $\nu>0$: for all $\ubf,\vbf\in \mathbf{V}$ it holds that
\begin{align}
\label{eq:bilinear_cont}
\left\{\begin{array}{l}
|a(\ubf,\ubf)|\geq c\frac{1}{4} \sqrt{\frac{\nu}{\heps + \theta^2}}\|u_2\|^2_{H_{1}(\Omega)} +  \frac{\nu}{4} \|\ubf\|_{L^2(\Omega)}^2,\\
|a(\ubf,\vbf)|\leq C \|\ubf\|_{\mathbf{V}}\|\vbf\|_{\mathbf{V}},
\end{array}\right.
\end{align}
where $c, C>0$ do not depend on $\nu$. The problem~\eqref{vf1dcase} is well-posed in $\mathbf{V}$.
\end{lemma}
%\textcolor{red}{!!!! redo coercivity computations !!!}
\begin{proof}
The boundedness of $a$ is obvious. Let us focus on the proof of coercivity. 
\be
\left\{\begin{array}{l}
a_1(\ubf,\vbf) = \int_{-L}^H (u_2' -\imath\theta u_1)\overline{(v_2' -\imath \theta v_1)} - \int_{-L}^H \eps_0 \ubf\cdot \overline{\vbf}, 
\\ a_2(\ubf,\vbf) = -\nu \int_{-L}^H  \ubf\cdot \overline{\vbf} - \lambda u_2 (-L) \overline{v_2 (-L)} , 
\end{array}\right.
\ee

\be
|a_2 (\ubf ,\ubf)| \geq \nu \int_{-L}^H \ubf\cdot \ubf
\ee
We note $\heps =  \|\rho(\eps_0)\|_{L^\infty}$, the spectral radius of $\eps_0$. Then the eigenvalues of $\heps Id- \eps_0$ are $\heps - \lambda_1$, $\heps - \lambda_2$, where $\lambda_1$ and $\lambda_2$ are the eigenvalues of $\eps_0$.

Then 
\be 
a_1(\ubf,\ubf) + \heps\|\ubf\|^2_{L^2}= \|u_2' - \imath \theta u_1 \|^2_{L^2} + \int_{-L}^{H} \left( (\heps Id - \eps_0 ) \ubf, \overline{\ubf} \right),
\ee
\be
a_1(\ubf,\ubf) + \heps\|\ubf \|  \geq \|u_2 ' - \imath \theta u_1 \|_{L^ 2}^ 2,
\ee
\be 
\begin{array}{l}\displaystyle
\|u_2' - \imath \theta u_1 \|_{L^2} = \|u_2'\|^2_{L^2} - 2 \Re \left(u_2', \overline{\imath \theta u_1}\right) + \theta^2 \|u_1\|^2_{L^2} \\
\geq \|u_2'\|_{L^2}^2 - 2 \left( \frac{1}{4}\|u_2' \|^2_{L^2} + \theta^2 \|u_1\|^2_{L^2} \right) + \theta^2 \|u_1\|^2_{L^2} \\
\geq \frac{1}{2}\| u_2' \|^2_{L^2}-  \theta^2 \|u_1 \|^2_{L^2}.
\end{array}
\ee
Thus, 
\be
a_1(\ubf,\ubf) + (\heps  + \theta^2) \| \ubf \|_{L^2}^2 \geq \frac{1}{2}\|u_2'\|^2_{L^2},
\ee
and
\be
\begin{array}{l}
|a(\ubf,\ubf)|^2 = |a_1(\ubf,\ubf)|^2 + |a_2(\ubf,\ubf)|^2 \\
\geq \left| \frac{1}{2}\|u_2' \|^2_{L^2} - (\heps + \theta^2) \|\ubf\|^2_{L^2}\right|^2+ \nu^2\|\ubf\|^2_{L^2}\\
\geq \frac{1}{4}\|u_2'\|_{L^2}^4 - (\heps + \theta^2) \|u_2'\|^2_{L^2} \|\ubf\|^2_{L^2} + \left( (\heps + \theta^2) + \nu \right) \| \ubf\|^4_{L^2}.
\end{array} 
\ee
%We obtain
%\be
%\begin{array}{l}
%| a(\ubf, \ubf) |^2 \geq \frac{1}{4} \| u_2'\|_{L^2}^2 - (\heps + \theta^2) \left( \frac{\sigma^2}{2}\|u_2'\|_{L^2}^4 \right) - (\heps + \theta^2) \left( \frac{\|\ubf\|^4_{L^2}}{2\sigma^2} \right) \\
%+ \left( (\heps + \sigma^2) + \nu^2 \right) \|\ubf \|^4_{L^2}\\
%\geq \left( \frac{1}{4}- \frac{(\heps + \theta^2)\sigma}{2}\right)\|u_2'\|_{L^2}^4 + \left(\left((\heps + \theta^2) + \nu^2\right) - \frac{(\heps + \theta^2)}{2\sigma^2}\right) \|\ubf\|_{L^2}^4
%\end{array}
For all $\sigma > 0$ we then have
\be\label{eq:coer01}
\begin{array}{l}
| a(\ubf, \ubf) |^2 \geq \frac{1}{4} \| u_2'\|_{L^2}^2 - (\heps + \theta^2) \left( \frac{\sigma^2}{2}\|u_2'\|_{L^2}^4 \right) - (\heps + \theta^2) \left( \frac{\|\ubf\|^4_{L^2}}{2\sigma^2} \right) \\
+ \left( (\heps + \theta^2) + \nu^2 \right) \|\ubf \|^4_{L^2}\\
\geq \left( \frac{1}{4}- \frac{(\heps + \theta^2)\sigma}{2}\right)\|u_2'\|_{L^2}^4 + \left(\left((\heps + \theta^2) + \nu^2\right) - \frac{(\heps + \theta^2)}{2\sigma^2}\right) \|\ubf\|_{L^2}^4.
\end{array}
\ee
%In order to satisfy \textcolor{green}{NOT VERY CLEAR }
%\be
%\frac{\sigma^2}{2} \leq \frac{1}{4(\heps + \theta^2)}.
%\ee
%and 
%\be
% \frac{\sigma^2}{2} \geq \frac{\heps+\theta^2}{4\left((\heps+\theta^2)^2+\nu^2\right)},
%\ee
%choose $\sigma$ such that 
Choosing
\be
\frac{\sigma^2}{2}= \sqrt{\frac{1}{4 (\heps+ \theta^2 )} \frac{\heps + \theta^2}{4((\heps + \theta^2)^2+ \nu^2)}},
\ee 
%so that
we obtain
\be
 \frac{1}{4}-\frac{\sigma^2}{2}(\heps - \theta^2)= \frac{1}{4}- \frac{1}{4}\frac{\heps + \theta^2}{\sqrt{(\heps + \theta^2)^2 + \nu^2}}
 \geq \frac{1}{4}\left(\frac{\frac{\nu}{\heps + \theta^2}}{1 + \frac{\nu}{\heps + \theta^2}}\right),
\ee
which, after Taylor expansion, gives
\be
\frac{\sigma^2}{2}\geq \frac{1}{4} \frac{\nu}{\heps+ \theta^2}.
\ee
Plugging the expression of $\sigma$ in \eqref{eq:coer01}, we get the following lower bound estimate
\be 
|a(\ubf,\ubf)|^2 \geq \frac{1}{4} \frac{\nu}{\heps + \theta^2}\|u_2'\|^4_{L^2} +  \frac{\nu^2}{2} \|\ubf\|_{L^2}^4,
\ee 
so that
\be 
|a(\ubf,\ubf)| \geq \frac{1}{4} \sqrt{\frac{\nu}{\heps + \theta^2}}\|u_2'\|^2_{L^2} +  \frac{\nu}{4} \|\ubf\|_{L^2}^2.
\ee 
Finally, due to Poincare's lemma, $a$ is coercive.


%\textcolor{blue}{
%Actually as 
%\be
% \| u - u(-L) \|_{L^2} \leq C_p \|u'\|_{L^2}, 
%\ee
%where $C_p = |\Omega|$,
%and 
%\be
%\begin{array}{l}
%\| u\|_{L^2 } = \| u - u(-L) + u(-L) \|_{L^2} \\
%\leq \| u - u(-L) \| + \sqrt{|\Omega|}|u(-L)|\\
%\leq |\Omega| \| u'\|_{L^2} + \sqrt{|\Omega|}|u(-L)|.
%\end{array} 
%\ee
%Let us recall that we have
%\be
%|a(u,u)| \geq\lambda |u(-L)|^2
%\ee
%and 
%\be
% |a(u,u)| \geq \frac{1}{4}\sqrt{\frac{\nu}{\heps + \theta^2}} \|u'\|^2_{L^2} +\frac{1}{4}\nu \|u\|_{L^2}^2. \label{eq:e1}
% \ee
%From \eqref{eq:e1} we get
% \be
% \begin{array}{l}
%  |a(u,u)| \geq \frac{1}{2}\sqrt{\alpha(-L)}|u(-L)|^2 + \frac{1}{8}\sqrt{\frac{\nu}{\heps + \theta^2}}\|u'\|_{L^2}^2 + \frac{1}{8}\nu \|u\|^2_{L^2} \\
%  \geq \left(\frac{1}{4}\sqrt{\alpha(-L)}|u(-L)|^2 + \frac{1}{16}\sqrt{\frac{\nu}{\heps + \theta^2}}\|u'\|^2_{L^2} + \frac{1}{16}\nu \|u\|^2 \right)\\
%  +\frac{1}{4}\sqrt{\alpha(-L)}|u(-L)|^2 + \frac{1}{16}\sqrt{\frac{\nu}{\heps + \theta^2}}\|u'\|^2,
% \end{array}
% \ee
%and 
%\be
%\begin{array}{l}
%\frac{1}{4}\sqrt{\alpha(-L)}|u(-L)|^2 + \frac{1}{16}\sqrt{\frac{\nu}{\heps + \theta^2}}\|u'\|_{L^2}^2\\
%= \frac{1}{4}\frac{\sqrt{\alpha(-L)}}{|\Omega|}\left(|{\Omega}| |u(-L)|^2\right) + \frac{1}{16}\frac{\sqrt{\frac{\nu}{\heps + \theta^2}}}{|\Omega|^2}\left(|\Omega|^2 \|u'\|^2\right)\\
%\geq \min\left(\frac{1}{4} \frac{\sqrt{\alpha(-L)}}{|\Omega|} , \frac{1}{16}\frac{\sqrt{\frac{\nu}{\heps + \theta^2}}}{|\Omega|^2}\right) \frac{1}{2}\|u\|^2_{L^2}
%\end{array}
%\ee
%then you obtain 
%\be 
%|a(u,u)| \geq \min\left(\frac{1}{4} \frac{\sqrt{\alpha(-L)}}{|\Omega|} , \frac{1}{16}\frac{\sqrt{\frac{\nu}{\heps + \theta^2}}}{|\Omega|^2}\right) \|u\|^2_{L^2}
%\ee
%}
Let us now examine the bicontinuity of $a$.
\be 
\begin{array}{l}
| a(\ubf, \vbf) |  \leq \left( \| u_2' \|_{L^2} + |\theta | \|u_1\|_{L^2}\right) \left( \| v_2' \|_{L^2} + |\theta | \|v_1\|_{L^2} \right)\\
+ (\heps+\nu) \| \ubf \|_{L^2} \| \vbf \|_{L^2} + |\alpha(-L) | |u_2(-L)| |v_2(-L)|
\\
\leq \vertiii{\ubf} \vertiii{\vbf} \left( 1 + 2|\theta| + \theta^2 + \heps + \nu + |\alpha(-L) | C_T^2 \right),
\end{array}
\ee  
where $\vertiii{\ubf} = \|u_2'\|_{L^2} + \|\ubf\|_{L^2}$ and $C_T$ the trace constant.

Suppose now $a(\ubf,\vbf) = l(\vbf)$, thus thanks to Lax-Milgram theorem we have existence and uniqueness of a variational solution.
\end{proof}

\begin{remark} \textcolor{green}{VERY UNCLEAR: WHY INTRODUCING M  IF WE NEVER USE IT ? CAN WE SAY WHEN IT ACTUALLY IS AN ISOMORPHISM ? FORWARD REFERENCE ?}
Let us assume that the operator
\begin{align*}
M\psi = \frac{i\delta}{(\alpha+i\nu)}\psi
\end{align*}
defines an isomorphism from $L_{2}(\Omega)$ to $L_{2}(\Omega)$. 
Given $E_x,\; E_y\in L_{2}(\Omega)$, 
the function $\mathcal{F}=\frac{i\delta}{(\alpha+\imath\nu)}\left((\alpha+\imath\nu)E_{x}+i\delta E_{y}\right)$ 
also belongs to $L_{2}(\Omega)$.  
If $E_x,\; E_y$ satisfy the problem (\ref{eq:var_form2}) for $\theta=0$, the function $\mathcal{F}=0$, 
i.e.
\begin{align*}
 (\mathcal{F},\tilde{E}_{y})_{L_{2}}=\int_{-L}^{H}\left(\imath\delta E_{x}-\delta^2(\alpha+i\nu)^{-1} E_{y}\right)\overline{\tilde E}_{y}=0,
\end{align*}
for all $\overline{\tilde E}_2\in H_{1}(\Omega)$. Adding this to the second expression of (\ref{eq:var_form2}), we obtain 
the variational formulation for $E_y$, which coincides with the variational formulation for the Helmholtz equation. The well-posedness 
of this problem had been studied in \cite{LMIG_thesis}. 
\end{remark}


%\subsection{2D case}
%%%%%%%%%%%%%%%%%%%%%% 


%%%%%%%%%%%%%%%%%%%%%% 
%\textcolor{red}{!!!!!!!!!!!!!!!! TODO : deal with this part !!!!!!!!!!!}
%$curl$ ibp formula
%\be
%\int_\Omega \curl u \cdot \f = \int_\Omega u\ curl \f + \int_\Gamma u (\f \wedge \n)
%\ee
%\be
%\int_\Omega curl \E \cdot \overline{curl \tilde \E} - \int_\Omega \eps \E \cdot \overline{\tilde \E} + \int _\Gamma curl \E \overline{\left( \tilde \E\wedge \n \right)} d\sigma = 0
%\ee
%\be
%\begin{array}{l}
%\displaystyle \int_\Omega curl \E \cdot \overline{curl \tilde \E} - \int_\Omega (\eps_0+i\nu Id) \E \cdot \overline{\tilde \E} - \int _{\{x=-L\}} \imath \sqrt{\alpha(-L)}(\E\wedge \n) \overline{\left( \tilde \E\wedge \n \right)} d\sigma 
%\\ \displaystyle \phantom{ fffffffff}= \int_{\{x=-L\}} g_{inc}  \overline{\left( \tilde \E\wedge \n \right)} d\sigma
%\end{array}\ee
%\textcolor{red}{!!!!!!!!!!!!!!!!!!!!!!!!!!!!!!!!!!!}


%%%%%%%%%%%%%%%%%%%%
\section{Discretization of the Frequency Domain Problem}
\label{sec:discr}
Testing the variational formulation (\ref{vf1dcase}) with $(\tilde E_x,0)$ and $(0,\; \tilde{E}_y)\in\mathbf{V}(\Omega)$, it becomes a system of two equations:
\begin{align*}
\begin{split}
i\theta \displaystyle \int_{-L}^H (E_y' -\imath\theta E_x)\overline{\tilde E_x} - \int_{-L}^H \left((\epsilon_0 +\imath\nu \operatorname{Id}) \E\right)_{1} \overline{\tilde E}_{x}
&=0,\\
\displaystyle \int_{-L}^H (E_y' -\imath\theta E_x)\tilde {E_y'} - \int_{-L}^H\left( (\epsilon_0 +\imath\nu\operatorname{Id}) \E\right)_{2}\overline{\tilde{E}}_{y}
  - \imath \lambda E_y (-L) \tilde E_y (-L) &= -g_{inc} (-L) \overline{( \tilde E_y(-L) )},\\
  \text{ for all } \tilde E_x\in L_{2}(\Omega), \; \tilde{E}_{y}\in H_{1}(\Omega).
  \end{split}
\end{align*}
Using the explicit  dielectric permittivity tensor into the above, it yields: for all $ \tilde E_x\in L_{2}(\Omega), \; \tilde{E}_{y}\in H_{1}(\Omega)$,
\begin{align}
\label{eq:var_form2}
\begin{split}
 i\theta \displaystyle \int_{-L}^H (E_y' -\imath\theta E_x)\overline{\tilde E_x} - 
 \int_{-L}^H \left((\alpha+i\nu)E_x+\imath\delta E_y\right)_{1} \overline{\tilde E}_{x}
&=0,\\
\int_{-L}^H (E_y' -\imath\theta E_x)\tilde {E_y'} -
\int_{-L}^H\left( -\imath \delta E_x+(\alpha+i\nu) E_y\right)_{2}\overline{\tilde{E}}_{y}
  - \imath \lambda E_y (-L) \tilde E_y (-L) &= -g_{inc} (-L) \overline{( \tilde E_y(-L) )}.
  \end{split}
\end{align}
Let us introduce two basis function spaces $V_{E_x}=\{\psi_{j}\}_{j=1}^{N_{1}}$ and $V_{E_{y}}=\{\phi_{i}\}_{i=1}^{N_{2}}$. 
We discretize problem (\ref{eq:var_form2}) as:
\begin{align*}
E_x=\sum\limits_{k=1}^{N_{1}}e_{xk}\psi_{k},\qquad E_{y}=\sum\limits_{k=1}^{N_{2}}e_{yk}\phi_{k}.
\end{align*}

% Substituting $\tilde{E}_{x}=\psi_{m}, \; m=1,\ldots,N_{1}$ and $\tilde{E}_{y}=0$ into the variational formulation (\ref{}), we obtain:
% \begin{align*}
% i\theta\sum\limits_{k=1}^{N_{2}}e_{2k}\int\limits_{-L}^{H}\phi'_{k}\bar{\psi}_{m}dx+\theta^2\sum\limits_{k=1}^{N_{1}}e_{1k}\int\limits_{-L}^{H}\psi_{k}\bar{\psi}_{m}dx\\
% -\sum\limits_{k=1}^{N_{1}}e_{1k}\int\limits_{-L}^{H}(\alpha(x)+i\nu)\psi_{k}\bar{\psi}_{m}dx-i\sum\limits_{1}^{N_{2}}e_{2k}\int\limits_{-L}^{H}\delta(x)\phi_{k}(x)\bar{\psi}_{m}dx=0.
% \end{align*}
% Similarly, for $\tilde{E}_{x}=0$ and $\tilde{E}_{y}=\phi_{\ell},\; \ell=1,\ldots, N_{2}$:
% \begin{align*}
% \sum\limits_{k=1}^{N_{2}}e_{2k}\int\limits_{-L}^{H}\phi'_{k}(x)\bar{\phi}'_{\ell}(x)dx-i\theta\sum\limits_{k=1}^{N_{1}}e_{1k}\int\limits_{-L}^{H}\psi_{k}(x)\bar{\phi}'_{\ell}(x)dx\\
% +i
% \sum\limits_{k=1}^{N_{1}}e_{1k}\int\limits_{-L}^{H}\delta(x)\psi_{k}\bar{\phi}_{m}dx-\sum\limits_{k=1}^{N_{2}}e_{2k}\int\limits_{-L}^{H}(\alpha(x)+i\nu)\phi_{k}\bar{\phi}_{m}dx\\
% -
% i\lambda\sum\limits_{k=1}^{N_{2}}e_{2k}\phi_{k}(-L)\bar{\phi}_{m}(-L)=-g_{inc}(x)\bar{\phi}_{m}(-L).
% \end{align*}
Consider the following notation:
\begin{align*}
\left(K^{\psi,\phi'}\right)_{mk}=\int\limits_{-L}^{H}\bar{\psi}_{m}\phi'_{k}dx,\qquad \left(M^{\psi}\right)_{mk}=\int\limits_{-L}^{H}\psi_{k}\bar{\psi}_{m}dx, \qquad 
\left(M^{\alpha,\psi}\right)_{mk}=\int\limits_{-L}^{H}(\alpha(x)+i\nu)\psi_{m}\bar{\psi}_{k}dx, \\
\left(M^{\delta,\psi,\phi}\right)_{mk}=\int\limits_{-L}^{H}\delta(x)\bar{\psi}_{m}\phi_{k}dx, \qquad 
K_{\ell k}=\int\limits_{-L}^{H}\phi'_{k}(x)\bar{\phi}'_{\ell}(x)dx,\qquad
\left(M^{\alpha,\phi}\right)_{\ell k}=\int\limits_{-L}^{H}(\alpha(x)+i\nu)\bar{\phi}_{\ell}\phi_{k}dx,\\
I_{km}^{\Gamma}=\bar{\phi}_{m}(-L)\phi_{k}(-L), 
\boldsymbol{e}_{x}=\left(e_{11},\ldots,e_{1 N_{1}}\right)^{T},\; \boldsymbol{e}_{y}=\left(e_{21},\ldots,e_{2 N_{1}}\right)^{T},\\
\boldsymbol{0}_{n} \text{ is an $n$-dimensional zero column vector}.
\end{align*}
Then system (\ref{eq:var_form2}) can be rewritten in an antisymmetric block form:
\begin{align*}
\left(\begin{matrix}
\theta^2 M^{\psi}-M^{\alpha,\psi} & i\theta K^{\psi,\phi'}-i M^{\delta,\psi,\phi} \\
-i\theta (K^{\psi,\phi'})^{*}+i (M^{\delta,\psi,\phi})^{*} & K-M^{\alpha,\phi}-i\lambda I^{\Gamma}
\end{matrix}\right)
\left(
\begin{matrix}
\boldsymbol{e}_x\\
\boldsymbol{e}_y
\end{matrix}
\right)=-g_{inc}(-L)
\left(
\begin{matrix}
\boldsymbol{0}_{N_{1}}\\
\bar{\phi}_{x}(-L)\\
\bar{\phi}_{y}(-L)\\
\vdots\\
\bar{\phi}_{N_{2}}(-L)
\end{matrix}
\right).
\end{align*}
This expression greatly simplifies when choosing $V_{E_x}=V_{E_{y}}=\left(\phi_{m}\right)_{m=1}^{N_{2}}$ and in the case $\theta=0$:
\begin{align}
\label{eq:simple_system}
\left(\begin{matrix}
M^{\alpha,\phi} & i M^{\delta,\phi,\phi} \\
i (M^{\delta,\phi,\phi})^{*} & K-M^{\alpha,\phi}-i\lambda I^{\Gamma}
\end{matrix}\right)
\left(
\begin{matrix}
\boldsymbol{e}_x\\ 
\boldsymbol{e}_y
\end{matrix}
\right)=-g_{inc}(-L)
\left(
\begin{matrix}
\boldsymbol{0}_{N_{1}}\\
\bar{\phi}_{x}(-L)\\
\bar{\phi}_{y}(-L)\\
\vdots\\
\bar{\phi}_{N_{2}}(-L)
\end{matrix}
\right).
\end{align}

\textcolor{green}{MENTION THE BASIS FUNCTIONS HERE ?}
