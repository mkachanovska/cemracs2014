La r\'esonnance hybride est un ph\'enom\`ene physique qui apparait par exemple lorsque l'on chauffe un plasma, et 
ainsi est d'int\'er\^et scientifique dans le cadre du d\'eveloppement  du projet ITER. Dans ce papier, nous nous concentrons sur certaines  solutions faiblement r\'eguli\`eres des \'equations de Maxwell pour les plasmas sous l'influence de champ magn\'etiques forts. Notre but est ici double. D'un c\^ot\'e nous \'evaluons l'approximation 
num\'erique \`a l'aide d'\'el\'ements finis en une dimension en formulation 
fr\'equentielle, et de l'autre nous \'etudions l'approximation num\'erique \`a l'aide de deux m\'ethodes de  diff\'erences finies pour la formulation
temporelle monodimensionelle. Nous comparons les r\'esultats de ces  diff\'erentes m\'ethodes.