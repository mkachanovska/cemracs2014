La r\'esonnance hybride est un ph\'enom\`ene physique qui apparait par exemple lorsque l'on chauffe un plasma, et est ainsi d'int\'er\^et dans le d\'eveloppement d'une nouvelle source d'\'energie, comme dans le cadre du projet ITER. Dans ce papier, nous nous concentrons sur la non r\'egularit\'e des solutions des \'equations de Maxwell pour les plasmas sous l'influence de champ magn\'etiques forts. Notre but est ici double. D'un c\^ot\'e nous regardons l'approximation \`a l'aide d'\'el\'ements finis du probl\`eme en une dimension \'ecrit dans le domaine en fr\'equence, et de l'autre nous regardons l'approximation \`a l'aide de diff\'erences finies du probl\`eme en une dimension mais d\'ependant en temps. Nous comparons \'egalement les r\'esultats de ces deux diff\'erentes m\'ethodes dans le contexte des principes d'absorption limite et de limitation d'amplitude.