\section{Discretization of the Frequency Domain Problem}
\label{sec:discr}
Testing the variational formulation (\ref{vf1dcase}) with $(\tilde E_x,0)$ and $(0,\; \tilde{E}_y)\in\mathbf{V}(\Omega)$, we can rewrite it as a system of two equations:
\begin{align*}
\begin{split}
i\theta \displaystyle \int_{-L}^H (E_y' -\imath\theta E_x)\overline{\tilde E_x} - \int_{-L}^H \left((\epsilon_0 +\imath\nu \operatorname{Id}) \E\right)_{1} \overline{\tilde E}_{x}
&=0,\\
\displaystyle \int_{-L}^H (E_y' -\imath\theta E_x)\tilde {E_y'} - \int_{-L}^H\left( (\epsilon_0 +\imath\nu\operatorname{Id}) \E\right)_{2}\overline{\tilde{E}}_{y}
  - \imath \lambda E_y (-L) \tilde E_y (-L) &= -g_{inc} (-L) \overline{( \tilde E_y(-L) )},\\
  \text{ for all } \tilde E_x\in L_{2}(\Omega), \; \tilde{E}_{y}\in H_{1}(\Omega).
  \end{split}
\end{align*}
Inserting an explicit expression for the dielectric permittivity tensor into the above, we obtain
\begin{align}
\label{eq:var_form2}
\begin{split}
 i\theta \displaystyle \int_{-L}^H (E_y' -\imath\theta E_x)\overline{\tilde E_x} - 
 \int_{-L}^H \left((\alpha+i\nu)E_x+\imath\delta E_y\right)_{1} \overline{\tilde E}_{x}
&=0,\\
\int_{-L}^H (E_y' -\imath\theta E_x)\tilde {E_y'} -
\int_{-L}^H\left( -\imath \delta E_x+(\alpha+i\nu) E_y\right)_{2}\overline{\tilde{E}}_{y}
  - \imath \lambda E_y (-L) \tilde E_y (-L) &= -g_{inc} (-L) \overline{( \tilde E_y(-L) )},\\
  \text{ for all } \tilde E_x\in L_{2}(\Omega), \; \tilde{E}_{y}\in H_{1}(\Omega).
  \end{split}
\end{align}
Let us introduce the basis spaces $V_{E_x}=\{\psi_{j}\}_{j=1}^{N_{1}}$ and $V_{E_{y}}=\{\phi_{i}\}_{i=1}^{N_{2}}$. 
We look for the solution of the problem (\ref{eq:var_form2}) in the form:
\begin{align*}
E_x=\sum\limits_{k=1}^{N_{1}}e_{xk}\psi_{k},\qquad E_{y}=\sum\limits_{k=1}^{N_{2}}e_{yk}\phi_{k}.
\end{align*}

% Substituting $\tilde{E}_{x}=\psi_{m}, \; m=1,\ldots,N_{1}$ and $\tilde{E}_{y}=0$ into the variational formulation (\ref{}), we obtain:
% \begin{align*}
% i\theta\sum\limits_{k=1}^{N_{2}}e_{2k}\int\limits_{-L}^{H}\phi'_{k}\bar{\psi}_{m}dx+\theta^2\sum\limits_{k=1}^{N_{1}}e_{1k}\int\limits_{-L}^{H}\psi_{k}\bar{\psi}_{m}dx\\
% -\sum\limits_{k=1}^{N_{1}}e_{1k}\int\limits_{-L}^{H}(\alpha(x)+i\nu)\psi_{k}\bar{\psi}_{m}dx-i\sum\limits_{1}^{N_{2}}e_{2k}\int\limits_{-L}^{H}\delta(x)\phi_{k}(x)\bar{\psi}_{m}dx=0.
% \end{align*}
% Similarly, for $\tilde{E}_{x}=0$ and $\tilde{E}_{y}=\phi_{\ell},\; \ell=1,\ldots, N_{2}$:
% \begin{align*}
% \sum\limits_{k=1}^{N_{2}}e_{2k}\int\limits_{-L}^{H}\phi'_{k}(x)\bar{\phi}'_{\ell}(x)dx-i\theta\sum\limits_{k=1}^{N_{1}}e_{1k}\int\limits_{-L}^{H}\psi_{k}(x)\bar{\phi}'_{\ell}(x)dx\\
% +i
% \sum\limits_{k=1}^{N_{1}}e_{1k}\int\limits_{-L}^{H}\delta(x)\psi_{k}\bar{\phi}_{m}dx-\sum\limits_{k=1}^{N_{2}}e_{2k}\int\limits_{-L}^{H}(\alpha(x)+i\nu)\phi_{k}\bar{\phi}_{m}dx\\
% -
% i\lambda\sum\limits_{k=1}^{N_{2}}e_{2k}\phi_{k}(-L)\bar{\phi}_{m}(-L)=-g_{inc}(x)\bar{\phi}_{m}(-L).
% \end{align*}
After introduction 
\begin{align*}
\left(K^{\psi,\phi'}\right)_{mk}=\int\limits_{-L}^{H}\bar{\psi}_{m}\phi'_{k}dx,\qquad \left(M^{\psi}\right)_{mk}=\int\limits_{-L}^{H}\psi_{k}\bar{\psi}_{m}dx, \qquad 
\left(M^{\alpha,\psi}\right)_{mk}=\int\limits_{-L}^{H}(\alpha(x)+i\nu)\psi_{m}\bar{\psi}_{k}dx, \\
\left(M^{\delta,\psi,\phi}\right)_{mk}=\int\limits_{-L}^{H}\delta(x)\bar{\psi}_{m}\phi_{k}dx, \qquad 
K_{\ell k}=\int\limits_{-L}^{H}\phi'_{k}(x)\bar{\phi}'_{\ell}(x)dx,\qquad
\left(M^{\alpha,\phi}\right)_{\ell k}=\int\limits_{-L}^{H}(\alpha(x)+i\nu)\bar{\phi}_{\ell}\phi_{k}dx,\\
I_{km}^{\Gamma}=\bar{\phi}_{m}(-L)\phi_{k}(-L), 
\boldsymbol{e}_{x}=\left(e_{11},\ldots,e_{1 N_{1}}\right)^{T},\; \boldsymbol{e}_{y}=\left(e_{21},\ldots,e_{2 N_{1}}\right)^{T},\\
\boldsymbol{0}_{n} \text{ is an $n$-dimensional zero column vector},
\end{align*}
the system (\ref{eq:var_form2}) can be rewritten in an antisymmetric block form:
\begin{align*}
\left(\begin{matrix}
\theta^2 M_{\psi}-M^{\alpha,\psi} & i\theta K^{\psi,\phi'}-i M^{\delta,\psi,\phi} \\
-i\theta (K^{\psi,\phi'})^{*}+i (M^{\delta,\psi,\phi})^{*} & K-M^{\alpha,\phi}-i\lambda I^{\Gamma}
\end{matrix}\right)
\left(
\begin{matrix}
\boldsymbol{e}_x\\
\boldsymbol{e}_y
\end{matrix}
\right)=-g_{inc}(-L)
\left(
\begin{matrix}
\boldsymbol{0}_{N_{1}}\\
\bar{\phi}_{x}(-L)\\
\bar{\phi}_{y}(-L)\\
\vdots\\
\bar{\phi}_{N_{2}}(-L)
\end{matrix}
\right).
\end{align*}
This expression greatly simplifies when choosing $V_{E_x}=V_{E_{y}}=\left(\phi_{m}\right)_{m=1}^{N_{2}}$ and in the case $\theta=0$:
\begin{align}
\label{eq:simple_system}
\left(\begin{matrix}
M^{\alpha,\phi} & i M^{\delta,\phi,\phi} \\
i (M^{\delta,\phi,\phi})^{*} & K-M^{\alpha,\phi}-i\lambda I^{\Gamma}
\end{matrix}\right)
\left(
\begin{matrix}
\boldsymbol{e}_x\\ 
\boldsymbol{e}_y
\end{matrix}
\right)=-g_{inc}(-L)
\left(
\begin{matrix}
\boldsymbol{0}_{N_{1}}\\
\bar{\phi}_{x}(-L)\\
\bar{\phi}_{y}(-L)\\
\vdots\\
\bar{\phi}_{N_{2}}(-L)
\end{matrix}
\right).
\end{align}
