

%%%%%%%%%%%%%%%%%%%%%% 
\section{1D case}
%%%%%%%%%%%%%%%%%%%%%% 


%%%%%%%%%%%%%%%%%%%%%% 
\subsection{Variational formulation, well posedness}
In one dimenssion, the variational formulation of the problem writes as
\be
\begin{array}{l}
\displaystyle \int_{-L}^H (E_2' -\imath\theta E_1)\overline{(\tilde E_2' -\imath \theta \tilde E_1)} - \int_{-L}^H (\eps_0 +\imath\nu Id) \E \cdot \overline{\tilde \E}
\\ \displaystyle  - \imath \sqrt{\alpha(-L)} E_2 (-L) \tilde E_2 (-L) = -g_{inc} (-L) \overline{( \tilde E_2(-L) )} \label{vf1dcase}.
\end{array}
\ee
From now on, we will use the following notations : 
\be
a(\ubf,\vbf) = a_1 (\ubf,\vbf) +\imath a_2(\ubf,\vbf)\  \text{ and } \  l(\vbf) = -g_{inc} (-L) \overline{(v_2(-L) )} 
\ee
where $a_1= a_1^*$ and $a_2=a_2^*$ are hermitian.
We now prove the following result.
\begin{lemma}
The problem~\eqref{vf1dcase} is well-posed.
\end{lemma}
\begin{proof}
\be
\left\{\begin{array}{l}
a_1(\ubf,\vbf) = \int_{-L}^H (u_2' -\imath\theta u_1)\overline{(v_2' -\imath \theta v_1)} - \int_{-L}^H \eps_0 \ubf\cdot \overline{\vbf}, 
\\ a_2(\ubf,\vbf) = -\nu \int_{-L}^H  \ubf\cdot \overline{\vbf} -  \sqrt{\alpha(-L)} u_2 (-L) \overline{v_2 (-L)} , 
\end{array}\right.
\ee

\be
|a_2 (\ubf ,\ubf)| \geq \nu \int_{-L}^H \ubf\cdot \ubf
\ee
We note $\heps =  \|\rho(\eps_0)\|_{L^\infty}$, the spectral radius of $\eps_0$. Then the eigenvalues of $\heps Id- \eps_0$ are $\heps - \lambda_1$, $\heps - \lambda_2$.

Then 
\be 
a_1(\ubf,\ubf) + \heps\|\ubf\|^2_{L^2}= \|u_2' - \imath \theta u_1 \|^2_{L^2} + \int_{-L}^{H} \left( (\heps Id - \eps_0 ) \ubf, \overline{\ubf} \right),
\ee
\be
a_1(\ubf,\ubf) + \heps\|\ubf \|  \geq \|u_2 ' - \imath \theta u_1 \|_{L^ 2}^ 2,
\ee
\be 
\begin{array}{l}\displaystyle
\|u_2' - \imath \theta u_1 \|_{L^2} = \|u_2'\|^2_{L^2} - 2 \Re \left(u_2', \overline{\imath \theta u_1}\right) + \theta^2 \|u_1\|^2_{L^2} \\
\geq \|u_2'\|_{L^2}^2 - 2 \left( \frac{1}{4}\|u_2' \|^2_{L^2} + \theta^2 \|u_1\|^2_{L^2} \right) + \theta^2 \|u_1\|^2_{L^2} \\
\geq \frac{1}{2}\| u_2' \|^2_{L^2}-  \theta^2 \|u_1 \|^2_{L^2}.
\end{array}
\ee
Thus, 
\be
a_1(\ubf,\ubf) + (\heps  + \theta^2) \| \ubf \|_{L^2}^2 \geq \frac{1}{2}\|u_2'\|^2_{L^2},
\ee
and
\be
\begin{array}{l}
|a(\ubf,\ubf)|^2 = |a_1(\ubf,\ubf)|^2 + |a_2(\ubf,\ubf)|^2 \\
\geq \left| \frac{1}{2}\|u_2' \|^2_{L^2} - (\heps + \theta^2) \|\ubf\|^2_{L^2}\right|^2+ \nu^2\|\ubf\|^2_{L^2}\\
\geq \frac{1}{4}\|u_2'\|_{L^2}^4 - (\heps + \theta^2) \|u_2'\|^2_{L^2} \|\ubf\|^2_{L^2} + \left( (\heps + \theta^2) + \nu \right) \| \ubf\|^4_{L^2}.
\end{array} 
\ee
%We obtain
%\be
%\begin{array}{l}
%| a(\ubf, \ubf) |^2 \geq \frac{1}{4} \| u_2'\|_{L^2}^2 - (\heps + \theta^2) \left( \frac{\sigma^2}{2}\|u_2'\|_{L^2}^4 \right) - (\heps + \theta^2) \left( \frac{\|\ubf\|^4_{L^2}}{2\sigma^2} \right) \\
%+ \left( (\heps + \sigma^2) + \nu^2 \right) \|\ubf \|^4_{L^2}\\
%\geq \left( \frac{1}{4}- \frac{(\heps + \theta^2)\sigma}{2}\right)\|u_2'\|_{L^2}^4 + \left(\left((\heps + \theta^2) + \nu^2\right) - \frac{(\heps + \theta^2)}{2\sigma^2}\right) \|\ubf\|_{L^2}^4
%\end{array}
For $\sigma > 0$ we have
\be
\begin{array}{l}
| a(\ubf, \ubf) |^2 \geq \frac{1}{4} \| u_2'\|_{L^2}^2 - (\heps + \theta^2) \left( \frac{\sigma^2}{2}\|u_2'\|_{L^2}^4 \right) - (\heps + \theta^2) \left( \frac{\|\ubf\|^4_{L^2}}{2\sigma^2} \right) \\
+ \left( (\heps + \theta^2) + \nu^2 \right) \|\ubf \|^4_{L^2}\\
\geq \left( \frac{1}{4}- \frac{(\heps + \theta^2)\sigma}{2}\right)\|u_2'\|_{L^2}^4 + \left(\left((\heps + \theta^2) + \nu^2\right) - \frac{(\heps + \theta^2)}{2\sigma^2}\right) \|\ubf\|_{L^2}^4.
\end{array}
\ee
We choose $\sigma$ such that 
\be
\frac{\sigma^2}{2} \leq \frac{1}{4(\heps + \theta^2)}.
\ee
and 
\be
 \frac{\sigma^2}{2} \geq \frac{\heps+\theta^2}{4\left((\heps+\theta^2)^2+\nu^2\right)}.
\ee
Taking then
\be
\frac{\sigma^2}{2}= \sqrt{\frac{1}{4 (\heps+ \theta^2 )} \frac{\heps + \theta^2}{4((\heps + \theta^2)^2+ \nu^2)}},
\ee 
and
\be
 \frac{1}{4}-\frac{\sigma^2}{2}(\heps - \theta^2)= \frac{1}{4}- \frac{1}{4}\frac{\heps + \theta^2}{\sqrt{(\heps + \theta^2)^2 + \nu^2}}
 \geq \frac{1}{4}\left(\frac{\frac{\nu}{\heps + \theta^2}}{1 + \frac{\nu}{\heps + \theta^2}}\right),
\ee
and using a limited development, gives, 
\be
\frac{\sigma^2}{2}\geq \frac{1}{4} \frac{\nu}{\heps+ \theta^2}.
\ee
Plugging the expression of $\sigma$ in $a$, we get the following minoration for 
\be 
|a(\ubf,\ubf)|^2 \geq \frac{1}{4} \frac{\nu}{\heps + \theta^2}\|u_2'\|^4_{L^2} +  \frac{\nu^2}{2} \|\ubf\|_{L^2}^4,
\ee 
meaning
\be 
|a(\ubf,\ubf)| \geq \frac{1}{4} \sqrt{\frac{\nu}{\heps + \theta^2}}\|u_2'\|^2_{L^2} +  \frac{\nu}{4} \|\ubf\|_{L^2}^2,
\ee 
and $a$ is coercive.
Actually as 
\be
 \| u - u(-L) \|_{L^2} \leq C_p \|u'\|_{L^2}, 
\ee
where $C_p = |\Omega|$,
and 
\be
\begin{array}{l}
\| u\|_{L^2 } = \| u - u(-L) + u(-L) \|_{L^2} \\
\leq \| u - u(-L) \| + \sqrt{|\Omega|}|u(-L)|\\
\leq |\Omega| \| u'\|_{L^2} + \sqrt{|\Omega|}|u(-L)|.
\end{array} 
\ee
Let us recall that we have
\be
|a(u,u)| \geq \sqrt{\alpha(-L)} |u(-L)|^2
\ee
and 
\be
 |a(u,u)| \geq \frac{1}{4}\sqrt{\frac{\nu}{\heps + \theta^2}} \|u'\|^2_{L^2} +\frac{1}{4}\nu \|u\|_{L^2}^2. \label{eq:e1}
 \ee
From \eqref{eq:e1} we get
 \be
 \begin{array}{l}
  |a(u,u)| \geq \frac{1}{2}\sqrt{\alpha(-L)}|u(-L)|^2 + \frac{1}{8}\sqrt{\frac{\nu}{\heps + \theta^2}}\|u'\|_{L^2}^2 + \frac{1}{8}\nu \|u\|^2_{L^2} \\
  \geq \left(\frac{1}{4}\sqrt{\alpha(-L)}|u(-L)|^2 + \frac{1}{16}\sqrt{\frac{\nu}{\heps + \theta^2}}\|u'\|^2_{L^2} + \frac{1}{16}\nu \|u\|^2 \right)\\
  +\frac{1}{4}\sqrt{\alpha(-L)}|u(-L)|^2 + \frac{1}{16}\sqrt{\frac{\nu}{\heps + \theta^2}}\|u'\|^2,
 \end{array}
 \ee
and 
\be
\begin{array}{l}
\frac{1}{4}\sqrt{\alpha(-L)}|u(-L)|^2 + \frac{1}{16}\sqrt{\frac{\nu}{\heps + \theta^2}}\|u'\|_{L^2}^2\\
= \frac{1}{4}\frac{\sqrt{\alpha(-L)}}{|\Omega|}\left(|{\Omega}| |u(-L)|^2\right) + \frac{1}{16}\frac{\sqrt{\frac{\nu}{\heps + \theta^2}}}{|\Omega|^2}\left(|\Omega|^2 \|u'\|^2\right)\\
\geq \min\left(\frac{1}{4} \frac{\sqrt{\alpha(-L)}}{|\Omega|} , \frac{1}{16}\frac{\sqrt{\frac{\nu}{\heps + \theta^2}}}{|\Omega|^2}\right) \frac{1}{2}\|u\|^2_{L^2}
\end{array}
\ee
then you obtain 
\be 
|a(u,u)| \geq \min\left(\frac{1}{4} \frac{\sqrt{\alpha(-L)}}{|\Omega|} , \frac{1}{16}\frac{\sqrt{\frac{\nu}{\heps + \theta^2}}}{|\Omega|^2}\right) \|u\|^2_{L^2}
\ee

Let us now examine the bicontinuity of $a$.
\be 
\begin{array}{l}
| a(\ubf, \vbf) |  \leq \left( \| u_2' \|_{L^2} + |\theta | \|u_1\|_{L^2}\right) \left( \| v_2' \|_{L^2} + |\theta | \|v_1\|_{L^2} \right)\\
+ (\heps+\nu) \| \ubf \|_{L^2} \| \vbf \|_{L^2} + |\alpha(-L) | |u_2(-L)| |v_2(-L)|
\\
\leq \vertiii{\ubf} \vertiii{\vbf} \left( 1 + 2|\theta| + \theta^2 + \heps + \nu + |\alpha(-L) | C_T^2 \right),
\end{array}
\ee  
where $\vertiii{\ubf} = \|u_2'\|_{L^2} + \|\ubf\|_{L^2}$ and $C_T$ the trace constant.

Suppose now $a(\ubf,\vbf) = l(\vbf)$, thus thanks to Lax-Milgram theorem we have existence and unicity of a variational solution.
\end{proof}
\section{2D case}
%%%%%%%%%%%%%%%%%%%%%% 


%%%%%%%%%%%%%%%%%%%%%% 
!!!!!!!!!!!!!!!! TODO : deal with this part !!!!!!!!!!!
\subsection{Variational formulation}
$curl$ ipp formula
\be
\int_\Omega \curl u \cdot \f = \int_\Omega u\ curl \f + \int_\Gamma u (\f \wedge \n)
\ee
\be
\int_\Omega curl \E \cdot \overline{curl \tilde \E} - \int_\Omega \eps \E \cdot \overline{\tilde \E} + \int _\Gamma curl \E \overline{\left( \tilde \E\wedge \n \right)} d\sigma = 0
\ee
\be
\begin{array}{l}
\displaystyle \int_\Omega curl \E \cdot \overline{curl \tilde \E} - \int_\Omega (\eps_0+i\nu Id) \E \cdot \overline{\tilde \E} - \int _{\{x=-L\}} \imath \sqrt{\alpha(-L)}(\E\wedge \n) \overline{\left( \tilde \E\wedge \n \right)} d\sigma 
\\ \displaystyle \phantom{ fffffffff}= \int_{\{x=-L\}} g_{inc}  \overline{\left( \tilde \E\wedge \n \right)} d\sigma
\end{array}\ee
!!!!!!!!!!!!!!!!!!!!!!!!!!!!!!!!!!!

