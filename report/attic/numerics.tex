\documentclass{article}
\usepackage{amsfonts,graphicx,amssymb}%amsfonts pour les \mathbb
\usepackage{amsmath,amsthm,amssymb}
\begin{document}
\section{Discretization}

\section{Numerical Experiments}
This section is organized as follows. The first part is dedicated to the numerical implementation
of the frequency domain formulation (\ref{}). We study the convergence of this formulation and the behaviour 
of the numerical solution as the absorption parameter $\nu$ tends to zero. The experiments in this section were 
performed on a laptop with ... CPU, with the help of the code written in Octave 
The second part of the section deals with the question of the equivalence of the limiting absorption and limited amplitude 
principle.
\subsection{Frequency Domain Problem}
This section is 



\subsection{Discretization of the Frequency Domain Problem}

Let us consider the case of the resonance, more precisely, we consider sufficiently smooth
$\alpha,\delta$, s.t. $\alpha(0)=0$ and $\delta(0)\neq 0$, and the solvability conditions 
of Theorem \ref{}  are satisfied. 
For simplicity, let us consider 
\begin{align}
\label{eq:cond}
 \alpha(x)=-x \text{  in some neighbourhood of $0$ }.
\end{align}


Given the space $\mathbf{V}_{h}=S_{h}^{1}\times S_{h}^{1}$, with $S_{h}^{1}$ consisting of piecewise-linear (hat) functions, we look for a ratio $h(\nu)$ that would ensure that the absolute error 
\begin{align}
\label{eq:problem1}
\|E^{\nu}_{1}-E^{\nu,h}_{1}\|_{L_{2}(\Omega)}<\epsilon,
\end{align}
given a fixed value of $\epsilon>0$. 

We use the following well-known facts:
\begin{itemize}
 \item The C\'ea's lemma applied to the problem (\ref{}); here $C_c$ is the continuity and $C_i$ is the coercivity constants:
\begin{align}
\label{eq:cea}
\begin{split}
 \|\mathbf{E}^{\nu}-\mathbf{E}^{h,\nu}\|_{V}\leq \frac{C_c}{C_i}\min_{\mathbf{v}\in V_h}\|\mathbf{E}-\mathbf{v}\|_{V}\\
 \leq \nu^{-1}\min_{\mathbf{v}\in V_h}\|\mathbf{E}-\mathbf{v}\|_{V}.
 \end{split}
\end{align}
The last inequality follows from Theorem \ref{}.  
\item The form of the exact solution to the problem (\ref{}):
\begin{align}
\label{eq:exact}
 E_{1}^{\nu}=E_{2}^{\nu}\frac{\delta}{\alpha+i\nu}=\frac{f(x)}{\alpha(x)+i\nu},
\end{align}
for some $f(x)\in L_{2}(\Omega)$.
\item The estimate from \cite[Chapter 0]{Brenner_Scott} on the rate of convergence of the interpolation 
\begin{align}
\label{eq:bsc}
 \|v-I^{h}v\|_{L_{2}(\Omega)}\leq Ch^2|v''|_{L_{2}(\Omega)},\; C>0,\\
 %\|v-I^{h}v\|_{H_{1}(\Omega)}\leq Ch|v''|_{L_{2}(\Omega)},\; C>0, 
\end{align}
where $I^{h}v$ is an interpolation operator onto $S_{h}^{1}$.
\end{itemize}
For $E_{1}^{\nu},\;f(x)$ in (\ref{eq:exact}) being sufficiently smooth, 
\begin{align*}
 \frac{d^2}{dx^2}E_{1}^{\nu}=\frac{f''}{\alpha+i\nu}-2\frac{f'\alpha'}{(\alpha+i\nu)^2}+\frac{f\alpha''}{(\alpha+i\nu)^3},
\end{align*}
from which, together with (\ref{eq:cond}), it follows that there exists $c>0$ s.t. for all sufficiently small $\nu$ 
\begin{align*}
 \left|  \frac{d^2}{dx^2}E_{1}^{\nu}\right|_{L_2}^{2}&\leq c\int\limits_{\Omega}\frac{1}{(x^2+\nu^2)^{3}}dx\\
 &=c\left.
 \left(\frac{x}{4(x^2+\nu^2)^2}\nu^{-2}+\frac{3x}{8(x^2+\nu^2)}\nu^{-4}+\frac{3}{8}\nu^{-5}\operatorname{atan}\frac{x}{\nu}\right)\right|_{-L}^{H}\\
 &\leq C\nu^{-5},\; 
\end{align*}
where $C>0$ does not depend on $\nu$. This, together with the estimates (\ref{eq:cea}) and (\ref{eq:bsc}), results in 
\begin{align*}
 \|E^{\nu}_{1}-E^{\nu,h}_{1}\|_{L_{2}(\Omega)}\leq C\nu^{-\frac{7}{2}}h^2,
\end{align*}
from which it follows that to ensure (\ref{eq:problem1}) $h$ should be chosen as $\alpha_{\epsilon}\nu^{\frac{7}{4}}$, 
where $\alpha_{\epsilon}>0$ depends on $\epsilon$ but does not depend on $\nu$. 

Let us check whether this holds true. 
To do so, we conduct the following numerical experiment.



We can show that 
\begin{align*}
 \|E^{\nu}_{1}\|_{L_{2}(\Omega)}\leq \frac{C}{\sqrt{\nu}},\; C>0, 
\end{align*}
and thus the relative error control
\begin{align*}
 \frac{\|E^{\nu}_{1}-E^{\nu,h}_{1}\|_{L_{2}(\Omega)}}{\|E^{\nu}_{1}\|_{L_{2}(\Omega)}}\leq \epsilon
\end{align*}
can be ensured by choosing $h$ as $\beta_{\epsilon}\nu^{\frac{3}{4}}$.

\subsection{Discretization of the Time Domain Problem}
\section{Numerical Experiments}
\subsection{Frequency Domain}
\subsection{Time Domain}
\end{document}