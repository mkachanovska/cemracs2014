\section{Conclusions}
In this work we considered the frequency- and the time-domain formulations of the $X$-mode Maxwell equations. 
In particular, we proved the well-posedness of the respective regularized (with the help of 
the absorption parameter $\nu$) variational formulation, as 
well as studied the convergence of the finite element method for the problem with the resonance. 
The piecewise-\urev{linear} FEM approximates the resonant solution rather well, at least for moderate values of $\nu$, 
however, the discretization size should be chosen roughly proportional to $\nu^{\frac{7}{4}}$, in order to obtain an accurate discretization.
Indeed, it would be interesting to look at the convergence of the adaptive finite elements for this kind of problems. 
Another unanswered question is the well-posedness of the discrete problem when the continuous problem is ill-posed. We have 
demonstrated by numerical means that while the condition number of the FEM matrix grows, the matrix remains always invertible, even for $\nu=0$. 

We proposed two different schemes for solving the time-dependent problem. 
Our numerical experiments demonstrate that both of them capture the singular behaviour of the solutions in the resonant case.

The other part of the experiments concerned the equivalence of the limiting absorption and limiting amplitude solutions. We have shown 
that for small $\nu$ and large times the solution computed in the time domain is close to the solution predicted by the limiting amplitude principle; as 
$\nu\rightarrow 0$, the solution oscillates harmonically as $t\rightarrow +\infty$, however, we were not able to compute the limiting amplitude solution for very 
small values of $\nu$. 