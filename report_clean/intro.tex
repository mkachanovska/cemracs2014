\section{Introduction}
Modelling various phenomena in plasmas is of practical importance for developing new sources of energy 
based on plasma fusion, see the ITER project\footnote{www.iter.org}. 
This article concentrates on studying a phenomenon of hybrid resonance \cite{Stix}, 
which is observed in experiments (see \cite{reflectometers_2006, reflectometers_2010, Dumont_2005}) and is described
mathematically as the non-regularity of
the solutions of Maxwell equations in plasmas under strong background magnetic field \cite{Despres_2014}. 
The energy deposit is resonant and may exceed by far the energy 
exchange which occurs in Landau damping, see \cite{Freidberg_2007}. 
Contrary to the Landau damping, however, 
hybrid resonance appears in a simpler model coupling 
fluid equations with the non electrostatic part of Maxwell equations.


\urev{
We consider the wave propagation in the cold plasma that consists of particles of single species 
under a uniform in time background magnetic field $\mathbf{B}_0=(0,\;0,\; B_0)$. This process in 2D can be described by the time-dependent Maxwell system  \cite{Stix}
\begin{align}
\label{eq:main_system}
\begin{split}
-\varepsilon_0 \partial_t E_x +\partial_y H_z = J_x,\\
-\varepsilon_0 \partial_t E_y -\partial_x H_z = J_y,\\
\mu_0 \partial_t H_z + \partial_x E_y-\partial_y E_x= 0
\end{split}
\end{align}
coupled with \mrev{an equation for the } linear electronic current  $\J = eN_e(\mathbf{r}) \ubf$, 
\begin{align}
\begin{split}
\label{eq:electronmove}
m_e \partial_t u_x =\epsilon_0 e E_x+e B_0(\mathbf{r}) u_y -m_e \nu u_x,\\
m_e \partial_t u_y =\epsilon_0 e E_y-e B_0(\mathbf{r}) u_x -m_e \nu u_y,
\end{split}
\end{align}
where \mrev{$\nu\geq 0$} is the friction  between particles, $e<0$ is the charge of electrons, $m_e$ the electron mass and $N_e$ the electron density that in general 
depends on spatial variables and is uniform in time. \mrev{A non-zero value of $\nu$ constitutes a basis for the limiting absorption principle.} 
The unknowns are the electron velocity $\ubf$ and the electromagnetic field. 


The energy of this system in a domain $\Omega\in\mathbb R^2$ in the case $\nu=0$ can be expressed as \cite{stable_yee_plasma_current}
\ben
{\mathcal E}(t)= \int_\Omega \left(
\frac{\epsilon_0 |\E(t,\mathbf{r})|^2}{2}+\frac{ |\B(t,\mathbf{r})|^2}{2\mu_0}+\frac{m_e|\J (t,\x)|^2}{2|e|N_e(\mathbf{r})} 
\right)\mathrm{d}\mathbf{r}.
\een

To illustrate unexpected difficulties that can occur in this system, let us rewrite it in the frequency domain. To do so, 
we introduce the plasma frequency $\omega_p(\mathbf{r})=\sqrt{\frac{|e|^2N_e(\mathbf{r})}{m\epsilon_0}}$, 
the cyclotron frequency $\omega_c(\mathbf{r})=\frac{e B_0(\mathbf{r})}{m_e}$ and perform the Fourier transform 
in time of \ref{eq:main_system} ($\partial_t\rightarrow -i\omega$), for a 
particular case $\nu=0$:
\begin{align}
\label{eq:model_freq_domain}
 \mathbf{curl}{\operatorname{curl}}\hat{\mathbf{E}}-\frac{\omega^2}{c^2}\underline{\epsilon}^{0}(\omega,\mathbf{r})\hat{\mathbf{E}}=0,
\end{align}
where the cold-plasma dielectric tensor \cite[Chapter 1-2]{Stix}
\begin{align*}
\underline{\epsilon}^{0}(\omega,\mathbf{r})=\left(
 \begin{matrix}
  \alpha(\mathbf{r}) & i\delta(\mathbf{r})\\
  -i\delta(\mathbf{r}) & \alpha(\mathbf{r})
 \end{matrix}
\right),\; \alpha(\mathbf{r})=1-\frac{\omega_p^2(\mathbf{r})}{\omega^2-\omega_c^2(\mathbf{r})},\; \delta=\frac{\omega_c(\mathbf{r})\omega_p^2(\mathbf{r})}{\omega(\omega^2-\omega_c^2(\mathbf{r}))}.
\end{align*}

The point $\mathbf{r}_c=(x_c,y_c)$ s.t. $\omega_c(\mathbf{r}_c)=\omega$, i.e. when $\underline{\epsilon}(\omega,\mathbf{r})$ is singular, is called a cyclotron resonance (c.f. \cite[Chapter 1-5]{Stix}). 
In \cite{singular_solutions} it was demonstrated for a 1D-counterpart of the above system that such points behave like removable singularities, 
and the solutions are well-defined in the limit $x\rightarrow x_c$. Hence this case is not of interest for the present work. 

Another interesting case is $\omega_p^2(\mathbf{r})+\omega_c^2(\mathbf{r})=\omega^2$, when the diagonal part of the tensor $\underline{\epsilon}(w,\mathbf{r})$ vanishes, 
under the condition that the non-diagonal part remains non-zero. This is a hybrid resonance case. 
As shown in \cite{Despres_2014, singular_solutions}, for a 1D-counterpart of (\ref{eq:model_freq_domain}), in this case the time-harmonic electric field component $E_x$ is not necessarily square 
integrable. This apparent paradox is of course the source of important numerical difficulties which are the subject of the present study.  
Jumping ahead, instead of (\ref{eq:model_freq_domain}) 
one considers a sequence of regularized problems
\begin{align}
\label{eq:seq_regularized}
  \mathbf{curl}{\operatorname{curl}}\hat{\mathbf{E}}-\frac{w^2}{c^2}\underline{\epsilon}^{\nu}(w,\mathbf{r})\hat{\mathbf{E}}=0,\;
  \underline{\epsilon}^{\nu}(\omega,\mathbf{r})= \underline{\epsilon}^{0}(\omega,\mathbf{r})+i\nu \mathbb{E},\; 
\end{align}
with $\nu\rightarrow 0$. 
}


\urev{In some physical applications (e.g. when modelling antennas in plasmas), it is required to study the above system in an 
infinite domain $(-L,\; \infty)\times\mathbb{R}$, $L\in \mathbb{R}$,
c.f. e.g. \cite{Despres_2014}. We complement Problem (\ref{eq:model_freq_domain}) with the boundary conditions.
At the left boundary of the domain, which represents the wall of the Tokamak, we choose Robin boundary conditions 
\be
-curl \hat{\mathbf{E}} -\imath \lambda\hat{\mathbf{E}} \wedge \n = g_{inc} = -curl \E_{inc} -\imath\lambda\E_{inc} \wedge \n,
\ee
where $\E_{inc} = \exp \left(\imath \lambda x\right)\begin{pmatrix} E_1\\ E_2 \end{pmatrix}$ and $\lambda$ is 
the frequency of the antenna. 


Therefore, we will concentrate on a 1D-version of the system (\ref{eq:main_model}), similarly to works \cite{Despres_2014, singular_solutions}. 
It already contains difficulties intrinsic to the hybrid resonance case. We truncate the domain 
$(-L,\; \infty)$ to $(-L,\; H)$ and set on the right boundary $\operatorname{curl} \E = 0$. }

Hence, we perform the Fourier transform in $y$ ($\partial_y\rightarrow i\theta$) and consider the one-dimensional setting

\begin{equation}
\label{eq:main_model}
\begin{cases}
\epsilon_0\partial_t E_{x}=-eN_e u_x,\\
\epsilon_0\partial_t E_{y}+\partial_x H_z=-eN_e u_y,\\
\mu_0\partial_t H_z+\partial_x E_y =0,\\
m_e\partial_t u_x=eE_x+eu_yB_0-\nu m_e u_x,\\
m_e\partial_t u_y=eE_y-eu_xB_0-\nu m_e u_y
\end{cases}
\end{equation}
 posed in $(-L,\; H)$, for some $L,H>0$, with the boundary conditions 
 \begin{align*}
  \left.\left(-\partial_x E_y+i\theta E_x -\imath \lambda E_y\right)\right|_{x=-L} =g_{inc},\\
  \left.\partial_xE_y\right|_{x=H}=0.
 \end{align*}

 %This domain represents the physical case of a wave sent from a wall facing a plasma. 
 
%In  \cite{Despres_2014} it was shown
%that the time harmonic electric field component $E_x$ in this case is not necessarily square 
%integrable, and  
%explicit estimates on the behavior of the solutions of \eqref{eq:main_model} in 1D were provided. 



%some given 
%frequency. 
%We truncate the domain 
%\ben
%-curl \E -\imath \lambda\E \wedge \n = g_{inc} = -curl \E_{inc} -\imath\lambda\E_{inc} \wedge \n,
%\een
%where $\E_{inc} = \exp \left(\imath \lambda x\right)\begin{pmatrix} E_1\\ E_2 \end{pmatrix}$.

%\begin{remark}
%	When modelling antennas in plasma physics \mrev{(e.g. in such applications as ITER)}, 
%	a good choice of boundary condition could be $\lambda = \sqrt{\alpha(-L)}$
%\end{remark}

The goal of this article is three-fold. 
First, we investigate the finite element approximation of the 1D 
problem \eqref{eq:main_model} written in the frequency domain ($\partial_t\rightarrow -i\omega$), with a dielectric tensor \mrev{regularized
with the help of a small parameter $\nu$}. More details on regularization will be given in further sections. 
%linearized with respect to $\nu$.
We prove the well-posedness of this problem for $\nu>0$ in Section~\ref{sec:wellposedness} and 
demonstrate that the use of  (P1) finite elements allows to approximate the singularity 
of the solution fairly well (Section \ref{sec:freq_dep}). 


Second, we briefly develop an original scheme based on widely appreciated semi-lagrangian schemes
for the discretization of time domain Maxwell's equations with a linear current.
Thirdly, we consider the case $\nu\rightarrow 0$, and study the limiting amplitude solution 
$\lim\limits_{t\rightarrow +\infty}\lim\limits_{\nu\rightarrow 0}\mathbf{E}(t)$ obtained with the help of 
the FDTD discretization of \eqref{eq:main_model}, suggested in \cite{stable_yee_plasma_current}. 
We compare this result with 
$\hat{\mathbf{E}}\mathrm{e}^{i\omega t}$, computed in the frequency domain, for $\nu\rightarrow 0$, which is equivalent to considering
$\lim\limits_{\nu\rightarrow 0}\lim\limits_{t\rightarrow+\infty}\mathbf{E}(t)$.
\textcolor{red}{Such a comparison is a way to study the formal commutation relation between the limited absorption and the limiting amplitude principle. 
In application to e.g. non-resonant case of (\ref{eq:model_freq_domain}), the limited absorption principle states that with 
appropriately chosen boundary conditions the solution of (\ref{eq:seq_regularized}) 
as $\nu\rightarrow 0\pm$ 
would coincide with the solution of (\ref{eq:model_freq_domain}). Similarly, the limiting amplitude principle applied to the problem (\ref{eq:main_system}) 
states that if the boundary conditions are chosen harmonic in time, e.g. $\partial_x E_y=\mathrm{e}^{-i\omega_{*} t}$ on 
the boundary, then the solution of this problem $(\mathbf{E}(t),\;H(t))$ asymptotically ($t\rightarrow +\infty$)
tends to a steady state of the form $ \left(\mathbf{E}(\omega_{*}),\; \hat{H}(\omega_{*})\right)\mathrm{e}^{-i\omega_{*} t}$, where 
$\left(\hat{\mathbf{E}}(\omega_{*}),\; \hat{H}(\omega_{*})\right)$ is the solution of (\ref{eq:main_model})
in the frequency domain, with $\omega=\omega_{*}$. The commutation between the two principles then write as}
$$
\lim\limits_{\nu\rightarrow 0}\lim\limits_{t\rightarrow+\infty}= \lim\limits_{t\rightarrow+\infty}\lim\limits_{\nu\rightarrow 0},
$$
see Figure \ref{fig:limits}. 
Even though it is true for standard linear wave problems, it is not clear \mrev{if it still holds} in our case due the singularity
of the hybrid resonance.
\begin{figure}
\begin{tikzpicture}
   \matrix (m) [matrix of math nodes,row sep=3em,column sep=4em,minimum width=2em] {
     \mathbf{E}^{\nu}(t) & \mathbf{E}^{0+}(t)\\
     \hat{\mathbf{E}}^{\nu}(t)\mathrm{e}^{i\omega t} & 
    \lim\limits_{\nu\rightarrow 0}\lim\limits_{t\rightarrow \infty}\mathbf{E}^{\nu}(t)=\lim\limits_{t\rightarrow \infty}\lim\limits_{\nu\rightarrow 0}\mathbf{E}^{\nu}(t)\\
     };
 \path[-stealth,font=\scriptsize]    (m-1-1) edge node [left] {$t\rightarrow+\infty$} (m-2-1);
  \path[-stealth, font=\scriptsize]   (m-1-1) edge node [above] {$\nu\rightarrow 0+$} (m-1-2);
 \path[-stealth, font=\scriptsize]   (m-2-1) edge node [above] {$\nu\rightarrow 0$} (m-2-2);
  \path[-stealth, font=\scriptsize]   (m-1-2) edge node [right] {$t\rightarrow +\infty$} (m-2-2);


  % \path[-stealth] (m-2-1.east) edge node [above] {$\nu\rightarrow 0$} (m-2-2);
   %(m-1-2) edge node [right] {$\nu\rightarrow 0$} (m-2-2);
\end{tikzpicture}
 \caption{Schematic representation of the equivalence of the limited absorption and limiting amplitude principles.}
\label{fig:limits}
\end{figure}

To our knowledge, such numerical studies have not been performed in the existing literature. 
