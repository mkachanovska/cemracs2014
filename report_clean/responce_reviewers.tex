\documentclass[a4paper,10pt]{article}
\usepackage[utf8]{inputenc}

%opening
\title{}
\author{}

\begin{document}

\maketitle

\begin{abstract}

\end{abstract}

\section{Extra Corrections}
As an extra improvement, not asked by the reviewers, we changed $\tilde{\omega}$ to $\omega_{\nu}$ in Section 2.

Corrected the definitions of $\alpha,\delta$ to comply with the definition of the cold plasma dielectric tensor of Stix (more precisely, removed terms $\frac{\omega^2}{c^2}$ in 
front of them)

\section{Responce to the reviewer #1}

All the corrections are marked in blue. 

A. General issues.

1. We agree with the remarks of the reviewer. We included a short introduction into the physics of the problem, 
which hopefully will facilitate in reading the article. The Section 4.3.1 contains of course a mistake in the description of the 
parameters of the problem. 

Concerning the notation: from now on $\epsilon^{\nu}_{\omega}(\mathbf{r})$ denotes the cold plasma dielectric tensor, similarly we denote its parameters. 


We added explicitely dependence on space in parameters that depend on the spatial variables. 

Indeed, $\epsilon_{\infty}<\infty$ if $\omega_c^2\neq \omega^2$, but formally one also needs to require the boundedness of $\omega_p^2$. Therefore we 
added an assumption in the end of the Section Frequency Domain Study on the coefficients of the dielectric tensor, and made a remark on the absence
of the cyclotron resonance. 


A brief description of what limiting absorption and limiting amplitude principles had been added in the end of the introduction.

2. The reduction to a 1d problem is not clearly justified - indeed, we changed this in the introduction by adding an assumption on $N_e, B_0$, 
and changed the wording in the introduction to underline that the 1D case is the one of interest. 


Specific remarks: 

- abstract: changed

- in the frequency domain, as we have shown in Lemma 2.1, the problem is regularized by a choice $|\nu|>0$, independently of the sign of $\nu$. 
It is interesting however that for a fixed $\omega$ the limits $\lim_{\nu\rightarrow 0\pm}$ do not result in the same solution. 
In the time domain the sign of $\nu>0$ is indeed of crucial importance for the stability. We added a remark on this before Section 2.1  

- $L$ is just a non-negative number. We found it more convenient to place the point of the isolated hybrid resonance in $x=0$.
For $\nu\neq 0$ the energy decays, we added this in the introduction
-
 



\end{document}
