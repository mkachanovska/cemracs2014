\documentclass[a4paper,10pt]{article}
\usepackage[utf8]{inputenc}

%opening
\title{}
\author{}

\begin{document}

\maketitle

\begin{abstract}

\end{abstract}

\section{General}

We marked corrections suggested by the reviewer 1 in blue, by the reviewer 2 in red, 
and some extra corrections made by us that can be important in green.

More important corrections include
- changed the wording in the abstract
- corrected the definitions of $\alpha,\delta$ to comply with the definition of the cold plasma dielectric tensor of Stix 
(more precisely, removed terms $\frac{\omega^2}{c^2}$ in 
front of them)
- corrected the formulation of the lemma on the well-posedness and added a remark after the lemma on the sign of $\nu$.



\section{Responce to the reviewer 1}

All the corrections are marked in blue. 

A. General issues.

1. We agree with the remarks of the reviewer. We included a short introduction into the physics of the problem, 
which hopefully will facilitate in reading the article. 

Concerning Sections 4.1.2, 4.3.1: 
We define more precisely the hybrid resonance now, which in fact requires that $\alpha$ vanishes, 
and $\delta$ does not (while in Section 4.3.1 $\delta$ vanishes as well).


The notation: 
from now on $\varepsilon^{\nu}_{\omega}(\mathbf{r})$ denotes the cold plasma dielectric tensor, similarly we denote its parameters. 
We added explicitely dependence on space in parameters that depend on the spatial variables. 

The case $\omega=\omega_c$:
Indeed, $\epsilon_{\infty}<\infty$ if $\omega_c^2\neq \omega^2$, but formally one also needs to require the boundedness of $\omega_p^2$. Therefore we 
added an assumption in the end of the Section Frequency Domain Study on the coefficients of the dielectric tensor, and made a remark on the absence
of the cyclotron resonance. 


The limiting amplitude/absorption principle: 

A brief description of what limiting absorption and limiting amplitude principles had been added in the end of the introduction.

2. 'The reduction to a 1d problem is not clearly justified' - indeed, we changed this in the introduction by adding an assumption on $N_e, B_0$, 
and changed the wording in the introduction to underline that the 1D case is the one of interest. 


Specific remarks: 

- p1, abstract: changed

- p2: in the frequency domain, as we have shown in Lemma 2.1, the problem is regularized by a choice $|\nu|>0$, independently of the sign of $\nu$. 
It is interesting however that for a fixed $\omega$ the limits $\lim_{\nu\rightarrow 0\pm}$ do not result in the same solution. 
In the time domain the sign of $\nu>0$ is indeed of crucial importance for the stability. We added a remark on this before Section 2.1  

- p2, after (1): $L$ is just a non-negative real number. We found it more convenient to place the point of the isolated hybrid resonance in $x=0$.
For $\nu\neq 0$ the energy decays, we added this in the introduction.


- p3: yes, indeed, fixed

- p3 eq (4): the notation had been changed
  
- p4, top: we changed the notation 

- p4, eq. (5) fixed

- p4, eq. (8): changed the notation in eq (8), though a more general result in Lemma 2.1 is formulated with $u,v$. 

- p4: we changed $\alpha_{\nu}$ to $\beta_{\nu}$ not to confuse with $\alpha$ in the dielectric tensor. The index $\nu$ 
just underlines the dependence of this angle $\beta_{\nu}$ on $\nu$ (indeed, the coercivity result as it is stated, holds for this specific value of $\beta_{\nu}$ stated 
in the end of the proof of the lemma).

- p. 5: indeed, we changed this 

- p.6:  changed

-p 6: $Ex$ changed

-p 8: our code, changed

 - p.8: fixed section number
 
 - p.8 eq (12): added $\nu=0$ + explanation
 
 - p.8: added the Airy equation and pointed out the analyticity of the Airy equation. Additionally added a remark on the behaviour of the Airy function to justify the choice of the right boundary 
 condition $\partial_x E=0$. 
 
 -  p9, Figure 3: fixed the captions. This had been done for other figures as well. The values $E^{c}_{x,y}$ had been defined in the caption.
 
 -p9 on top
  The title of Section 4.1.2 had been changed to Solution of a Frequency-Domain Problem with Resonance
  
 - p9 Equation (15,16): changed  

 - p9 after (17): we would like to assume that it is $H^2$-norm that is bounded, for using the inequality (19). 
  \mrev{TODO: While this assumption is fairly strong, it is often used in the approximation theory.   }
  
 - p10 before and after Table 1 : fixed
 
 -- p11 Figure 5:
Agree, fixed. We added a remark in a caption to Figure 4 as well.



- p12 Section 4.3: this explanation had been moved to the introduction



- p12 Section 4.3.1

  It is true, but in this case there is no resonance, since $\delta(x)\equiv 0$. Hence $E_x=-\frac{i\delta}{\alpha}E_y\equiv 0$, 
  and $E_y$ solves the Airy equation in the frequency domain. We added a small to Section 4.3.1 to underline this. 

  
- p12 Equation (23)
We changed it to $e$
  
- p14 Figures 8 and 9
Added the captions  
- p15 in Conclusion
  Indeed $P1$!
  

 Reviewer 2:
 
 • p2: In this case Ex is not square integrable: confirm that it is when ν is not zero and that
the singularity appears for ν = 0.

We tried to change the wording in the introduction to make it more clear what is a hybrid resonance. 
The singularity appears for $\nu=0$, when the diagonal part of the dielectric tensor vanishes and non-diagonal part does not. 

• p6, 3.1. It is not clear how the Yee scheme, which is a finite difference scheme, is linked to the
finite element scheme developed in the previous section. This should be made more explicit.

We employed the Yee scheme to discretize the problem in the time domain, and the finite element scheme solely for the frequency domain, 
and considered them independently. We added a remark on that in the very beginning of Section 3. 

• p10: mesh multiple times finer. It should be said exactly how many times. 

We added a remark that it was at least two times for very fine meshes, and more for coarser ones.


  
 



\end{document}
