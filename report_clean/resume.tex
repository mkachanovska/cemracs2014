La r\'esonnance hybride est un ph\'enom\`ene physique qui apparait par exemple lorsque l'on chauffe un plasma, et 
ainsi est d'int\'er\^et scientifique dans le cadre du d\'eveloppement  du projet ITER. Dans ce papier, nous nous concentrons sur des solutions faiblement r\'eguli\`eres des \'equations de Maxwell pour les plasmas sous l'influence de champ magn\'etiques forts. Notre but principal est ici double. D'un c\^ot\'e nous \'evaluons l'approximation 
num\'erique \`a l'aide d'\'el\'ements finis de la formulation 
fr\'equentielle, et de l'autre nous \'etudions les solutions r\'esonnantes dans le domaine temporelle, \`a l'aide de deux approximation diff\'erences finies diff\'erentes. Nous comparons ensuite les r\'esultats ainsi obtenus avec ceux donn\'es dans le domaine fr\'equentiel. Ceci est fait en regardant num\'eriquement les principes d'amplitude et d'absorption limite.


