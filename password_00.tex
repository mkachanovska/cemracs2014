\documentclass{article}




\usepackage{amsfonts,graphicx,amssymb}%amsfonts pour les \mathbb
\usepackage{amsmath,amsthm,amssymb}


\newcommand{\bs}{\left\{}
\newcommand{\es}{\right.}
\newcommand{\ba}{\begin{array}}
\newcommand{\ea}{\end{array}}
\newcommand{\be}{\begin{equation}}
\newcommand{\ee}{\end{equation}}
\newcommand{\ora}{\overrightarrow}
\newcommand{\x}{{\bf x}}
\newcommand{\E}{{\bf E}}
\newcommand{\n}{{\bf n}}
\newcommand{\f}{{\bf f}}

\newcommand{\ubf}{{\bf u}}
\newcommand{\vbf}{{\bf v}}
\newcommand{\curl}{{\bf curl}}
\newcommand{\eps}{{\bf \varepsilon}}
\newcommand{\heps}{{\hat{\eps}}}
\def\RR{\mathbf R }
\newcommand{\vertiii}[1]{{\left\vert\kern-0.25ex\left\vert\kern-0.25ex\left\vert #1 
    \right\vert\kern-0.25ex\right\vert\kern-0.25ex\right\vert}}
\newtheorem{theor}{Theorem}
\newtheorem{Lemma}{Lemma}
\newtheorem{rem}{Remark}
\newtheorem{assumption}{Assumption}



\title{PASSWORD - CEMRACS 2014}
\author{Maryna, C\'eline, Bruno \& LM}
\date{}
\begin{document}
\maketitle


%%%%%%%%%%%%%%%%%%%%%% 
\section{Introduction}
%%%%%%%%%%%%%%%%%%%%%% 


\begin{equation} \label{eq:00}
\bs
\ba{ccccc}
-\partial^2_{y^2}E_{1}^\nu & +\partial^2_{xy}E_{2}^\nu & -\epsilon_{11}^\nu
 E_{1}^\nu & -\epsilon_{12}^\nu E_{2}^\nu & =0, \cr
\partial^2_{xy}E_{1}^\nu & -\partial^2_{xx}E_{2}^\nu & -\epsilon_{21}
^\nu
E_{1}^\nu & -\epsilon_{22}^\nu E_{2}^\nu & =0,
\ea
\es
\end{equation}
\be \label{eq:epsilonmu}
 \epsilon^\nu(x) =
\left(
\begin{array}{cc}
\epsilon_{11} ^\nu(x)& \epsilon_{12} ^\nu(x)\\
\epsilon_{21}^\nu(x) & \epsilon_{22}^\nu(x)
\end{array}
\right)
\in \mathbb C^{2\times 2}
.
\ee

2D model
\be
{\bf curl} \ curl \E - \frac{\omega^2}{c^2} \varepsilon \E= 0
\ee
\be
\eps (\x)= 
\left(
\begin{array}{cc}
\alpha(\x) + \imath \nu& \imath \delta (\x)\\
-\imath \delta (\x) & \alpha(\x) +\imath \nu
\end{array}
\right)
\ee

Domain $\Omega$, boundary $\Gamma$

%%%%%%%%%%%%%%%%%%%%%% 
\subsection{Boundary condition}
Robin Boundary condition at the left boundary of the domain
\be
-curl \E -\imath \sqrt{\alpha(-{ L})}\E \wedge \n = g_{inc} = -curl \E_{inc} -\imath \sqrt{\alpha(-{ L})}\E_{inc} \wedge \n,
\ee
where  $\E_{inc} = \exp \left(\imath \sqrt{\alpha(-L)}x\right)\begin{pmatrix} E_1\\ E_2 \end{pmatrix}$

$curl \E = 0$ at the right boundary


%%%%%%%%%%%%%%%%%%%%%% 
\section{1D case}
%%%%%%%%%%%%%%%%%%%%%% 


%%%%%%%%%%%%%%%%%%%%%% 
\subsection{Variational formulation, well posedness}
1D: $\partial_y = \imath \theta$
\be
\begin{array}{l}
\displaystyle \int_{-L}^H (E_2' -\imath\theta E_1)\overline{(\tilde E_2' -\imath \theta \tilde E_1)} - \int_{-L}^H (\eps_0 +\imath\nu Id) \E \cdot \overline{\tilde \E}
\\ \displaystyle  - \imath \sqrt{\alpha(-L)} E_2 (-L) \tilde E_2 (-L) = -g_{inc} (-L) \overline{( \tilde E_2(-L) )} 
\end{array}
\ee
Notation
\be
a(\ubf,\vbf) = a_1 (\ubf,\vbf) +\imath a_2(\ubf,\vbf)\  \text{ and } \  l(\vbf) = -g_{inc} (-L) \overline{(v_2(-L) )} 
\ee
\be
\left\{\begin{array}{l}
a_1(\ubf,\vbf) = \int_{-L}^H (u_2' -\imath\theta u_1)\overline{(v_2' -\imath \theta v_1)} - \int_{-L}^H \eps_0 \ubf\cdot \overline{\vbf}, 
\\ a_2(\ubf,\vbf) = -\nu \int_{-L}^H  \ubf\cdot \overline{\vbf} -  \sqrt{\alpha(-L)} u_2 (-L) \overline{v_2 (-L)} , 
\end{array}\right.
\ee
where $a_1= a_1^*$ and $a_2=a_2^*$ are hermitian.
\be
|a_2 (\ubf ,\ubf)| \geq \nu \int_{-L}^H \ubf\cdot \ubf
\ee
We note $\heps =  \|\rho(\eps_0)\|_{L^\infty}$, the spectral radius of $\eps_0$. Then the eigenvalues of $\heps Id- \eps_0$ are $\heps - \lambda_1$, $\heps - \lambda_2$.

Then 
\be 
a_1(\ubf,\ubf) + \heps\|\ubf\|^2_{L^2}= \|u_2' - \imath \theta u_1 \|^2_{L^2} + \int_{-L}^{H} \left( (\heps Id - \eps_0 ) \ubf, \overline{\ubf} \right),
\ee
\be
a_1(\ubf,\ubf) + \heps\|\ubf \|  \geq \|u_2 ' - \imath \theta u_1 \|_{L^ 2}^ 2,
\ee
\be 
\begin{array}{l}\displaystyle
\|u_2' - \imath \theta u_1 \|_{L^2} = \|u_2'\|^2_{L^2} - 2 \Re \left(u_2', \overline{\imath \theta u_1}\right) + \theta^2 \|u_1\|^2_{L^2} \\
\geq \|u_2'\|_{L^2}^2 - 2 \left( \frac{1}{4}\|u_2' \|^2_{L^2} + \theta^2 \|u_1\|^2_{L^2} \right) + \theta^2 \|u_1\|^2_{L^2} \\
\geq \frac{1}{2}\| u_2' \|^2_{L^2}-  \theta^2 \|u_1 \|^2_{L^2}.
\end{array}
\ee
Thus, 
\be
a_1(\ubf,\ubf) + (\heps  + \theta^2) \| \ubf \|_{L^2}^2 \geq \frac{1}{2}\|u_2'\|^2_{L^2},
\ee
and
\be
\begin{array}{l}
|a(\ubf,\ubf)|^2 = |a_1(\ubf,\ubf)|^2 + |a_2(\ubf,\ubf)|^2 \\
\geq \left| \frac{1}{2}\|u_2' \|^2_{L^2} - (\heps + \theta^2) \|\ubf\|^2_{L^2}\right|^2+ \nu^2\|\ubf\|^2_{L^2}\\
\geq \frac{1}{4}\|u_2'\|_{L^2}^4 - (\heps + \theta^2) \|u_2'\|^2_{L^2} \|\ubf\|^2_{L^2} + \left( (\heps + \theta^2) + \nu \right) \| \ubf\|^4_{L^2}.
\end{array} 
\ee
We obtain
\be
\begin{array}{l}
| a(\ubf, \ubf) |^2 \geq \frac{1}{4} \| u_2'\|_{L^2}^2 - (\heps + \theta^2) \left( \frac{\sigma^2}{2}\|u_2'\|_{L^2}^4 \right) - (\heps + \theta^2) \left( \frac{\|\ubf\|^4_{L^2}}{2\sigma^2} \right) \\
+ \left( (\heps + \sigma^2) + \nu^2 \right) \|\ubf \|^4_{L^2}\\
\geq \left( \frac{1}{4}- \frac{(\heps + \theta^2)\sigma}{2}\right)\|u_2'\|_{L^2}^4 + \left(\left((\heps + \theta^2) + \nu^2\right) - \frac{(\heps + \theta^2)}{2\sigma^2}\right) \|\ubf\|_{L^2}^4
\end{array}
\ee
We choose $\sigma$ such that 
\be
\frac{\sigma^2}{2} \leq \frac{1}{4(\heps + \theta^2)}.
\ee
and 
\be
 \frac{\sigma^2}{2} \geq \frac{\heps+\theta^2}{4\left((\heps+\theta^2)^2+\nu^2\right)}
\ee
We  take
\be
\frac{\sigma^2}{2}= \sqrt{\frac{1}{4 (\heps+ \theta^2 )} \frac{\heps + \theta^2}{4((\heps + \theta^2)^2+ \nu^2)}},
\ee 
then 
\be
 \frac{1}{4}-\frac{\sigma^2}{2}(\heps - \theta^2)= \frac{1}{4}- \frac{1}{4}\frac{\heps + \theta^2}{\sqrt{(\heps + \theta^2)^2 + \nu^2}}
 \geq \frac{1}{4}\left(\frac{\frac{\nu}{\heps + \theta^2}}{1 + \frac{\nu}{\heps + \theta^2}}\right),
\ee
which, using a limited development, gives, 
\be
\frac{\sigma^2}{2}\geq \frac{1}{4} \frac{\nu}{\heps+ \theta^2}
\ee
Putting the expression of $\sigma$ in $a$, we get the following minoration for 
\be 
|a(\ubf,\ubf)|^2 \geq \frac{1}{4} \frac{\nu}{\heps + \theta^2}\|u_2'\|^4_{L^2} +  \frac{\nu^2}{2} \|\ubf\|_{L^2}^4,
\ee 
meaning
\be 
|a(\ubf,\ubf)| \geq \frac{1}{4} \sqrt{\frac{\nu}{\heps + \theta^2}}\|u_2'\|^2_{L^2} +  \frac{\nu}{4} \|\ubf\|_{L^2}^2,
\ee 
and $a$ is coercive.
Actually as 
\be
 \| u - u(-L) \|_{L^2} \leq C_p \|u'\|_{L^2}, 
\ee
where $C_p = |\Omega|$,
and 
\be
\begin{array}{l}
\| u\|_{L^2 } = \| u - u(-L) + u(-L) \|_{L^2} \\
\leq \| u - u(-L) \| + \sqrt{|\Omega|}|u(-L)|\\
\leq |\Omega| \| u'\|_{L^2} + \sqrt{|\Omega|}|u(-L)|.
\end{array} 
\ee
Let us recall that we have
\be
|a(u,u)| \geq \sqrt{\alpha(-L)} |u(-L)|^2
\ee
and 
\be
 |a(u,u)| \geq \frac{1}{4}\sqrt{\frac{\nu}{\heps + \theta^2}} \|u'\|^2_{L^2} +\frac{1}{4}\nu \|u\|_{L^2}^2. \label{eq:e1}
 \ee
From \eqref{eq:e1} we get
 \be
 \begin{array}{l}
  |a(u,u)| \geq \frac{1}{2}\sqrt{\alpha(-L)}|u(-L)|^2 + \frac{1}{8}\sqrt{\frac{\nu}{\heps + \theta^2}}\|u'\|_{L^2}^2 + \frac{1}{8}\nu \|u\|^2 \\
  \geq \left(\frac{1}{4}\sqrt{\alpha(-L)}|u(-L)|^2 + \frac{1}{16}\sqrt{\frac{\nu}{\heps + \theta^2}}\|u\|^2_{L^2} + \frac{1}{16}\nu \|u\|^2 \right)\\
  +\frac{1}{4}\sqrt{\alpha(-L)}|u(-L)|^2 + \frac{1}{16}\sqrt{\frac{\nu}{\heps + \theta^2}}\|u'\|^2,
 \end{array}
 \ee
and 
\be
\begin{array}{l}
\frac{1}{4}\sqrt{\alpha(-L)}|u(-L)|^2 + \frac{1}{16}\sqrt{\frac{\nu}{\heps + \theta^2}}\|u'\|_{L^2}^2\\
= \frac{1}{4}\frac{\sqrt{\alpha(-L)}}{\sqrt{|\Omega|}}\left(\sqrt{\Omega}|u(-L)|^2\right) + \frac{1}{16}\frac{\sqrt{\frac{\nu}{\heps + \theta^2}}}{|\Omega|^2}\left(|\Omega| \|u'\|^2\right)\\
\geq \min\left(\frac{1}{4} \frac{\sqrt{\alpha(-L)}}{|\Omega|} , \frac{1}{16}\frac{\sqrt{\frac{\nu}{\heps + \theta^2}}}{|\Omega|^2}\right) \frac{1}{2}\|u\|^2_{L^2}
\end{array}
\ee
Let us now examine the bicontinuity of $a$.
\be 
\begin{array}{l}
| a(\ubf, \vbf) |  \leq \left( \| u_2' \|_{L^2} + |\theta | \|u_1\|_{L^2}\right) \left( \| v_2' \|_{L^2} + |\theta | \|v_1\|_{L^2} \right)\\
+ (\heps+\nu) \| \ubf \|_{L^2} \| \vbf \|_{L^2} + |\alpha(-L) | |u_2(-L)| |v_2(-L)|
\\
\leq \vertiii{\ubf} \vertiii{\vbf} \left( 1 + 2|\theta| + \theta^2 + \heps + \nu + |\alpha(-L) | C_T^2 \right),
\end{array}
\ee  
where $\vertiii{\ubf} = \|u_2'\|_{L^2} + \|\ubf\|_{L^2}$ and $C_T$ the trace constant.

Suppose now $a(\ubf,\vbf) = l(\vbf)$, thanks to Lax-Milgram theorem we have existence and unicity of a variational solution.
%%%%%%%%%%%%%%%%%%%%%% 
\section{2D case}
%%%%%%%%%%%%%%%%%%%%%% 


%%%%%%%%%%%%%%%%%%%%%% 
\subsection{Variational formulation}
$curl$ ipp formula
\be
\int_\Omega \curl u \cdot \f = \int_\Omega u\ curl \f + \int_\Gamma u (\f \wedge \n)
\ee
\be
\int_\Omega curl \E \cdot \overline{curl \tilde \E} - \int_\Omega \eps \E \cdot \overline{\tilde \E} + \int _\Gamma curl \E \overline{\left( \tilde \E\wedge \n \right)} d\sigma = 0
\ee
\be
\begin{array}{l}
\displaystyle \int_\Omega curl \E \cdot \overline{curl \tilde \E} - \int_\Omega (\eps_0+i\nu Id) \E \cdot \overline{\tilde \E} - \int _{\{x=-L\}} \imath \sqrt{\alpha(-L)}(\E\wedge \n) \overline{\left( \tilde \E\wedge \n \right)} d\sigma 
\\ \displaystyle \phantom{ fffffffff}= \int_{\{x=-L\}} g_{inc}  \overline{\left( \tilde \E\wedge \n \right)} d\sigma
\end{array}\ee


\begin{thebibliography}{99}


\bibitem{CW} %Chen-White
F. F. Chen and R. B. White.
Amplification and Absorption of 
Electromagnetic Waves in Overdense Plasmas, Plasma Phys. 16565
 (1974); anthologized in Laser Interaction with Matter, 
Series of Selected Papers in Physics, ed. by C. Yamanaka,  Phys. Soc. Japan, 1984



\bibitem{co:ed} E. A. Coddington and N. Levinson,
Theory of ordinary differential equations, 
McGraw-Hill, 1955 


\bibitem{jmpa} B. Despres, L.-M. Imbert-Gerard and R. Weder, 
Hybrid resonance of Maxwell's equations in slab geometry, 
Journal de Mathématiques Pures et Appliquées
Volume 101, Issue 5, May 2014, Pages 623?659.



\bibitem{LWZ} O. Lafitte, M. Williams and K. Zumbrun,
High-frequency stability of multidimensional ZND detonations, 2013,
arxiv preprint http://arxiv.org/abs/1312.6906.


\bibitem{stix} T. H. Stix,
The Theory of Plasma Waves, 1962 , New York: McGraw-Hill. 

\bibitem{watson} D.  G. Swanson,
Plasma Waves, 2nd Edition, 2003,
Series in Plasma Physics.



\end{thebibliography}

\end{document}