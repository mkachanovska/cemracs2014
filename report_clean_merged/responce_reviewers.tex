\documentclass[a4paper,10pt]{article}
\usepackage[utf8]{inputenc}
\usepackage{color}

%opening
\title{}
\author{}

\begin{document}

\maketitle

\begin{abstract}

\end{abstract}

\section{General}

We marked corrections suggested by the reviewer 1 in blue, by the reviewer 2 in red, 
and some extra corrections made by us that can be important in green.

More important corrections include
\begin{itemize}
\item changed the wording in the abstract

\item corrected the definition of the cold plasma dielectric tensor, to match the convention $\partial_t\rightarrow -i\omega$ (it had to be hermitian conjugate; 
this had been done everywhere, the corresponding corrections had been made in Table 1 and some figures)

\item corrected the definitions of $\alpha,\delta$ to comply with the definition of the cold plasma dielectric tensor of Stix 

(more precisely, removed terms $\frac{\omega^2}{c^2}$ in 
front of them)

\item corrected the formulation of the lemma 2.3 on the well-posedness and added a remark after the lemma on the sign of $\nu$.

\item Section 4.1.1: corrected assumption on smoothness of $E_y^{\nu}$ in the section 'Numerical Experiments'.

\item shortened slightly the section on limiting amplitude for a non-resonant case.
\end{itemize}


\section{Response to the reviewer 1}

All the corrections are marked in blue. 

A. General issues.

1. We agree with the remarks of the reviewer. We included a short introduction into the physics of the problem, 
which hopefully will facilitate in reading the article. 
\begin{itemize}
 \item Concerning Sections 4.1.2, 4.3.1: 
We define more precisely the hybrid resonance now, which in fact requires that $\alpha$ vanishes, 
and $\delta$ does not (while in Section 4.3.1 $\delta$ vanishes as well).
To give a hint on singularity of equations, we cite an article where a simple model problem is considered. 

\item 
The notation: 
from now on $\varepsilon^{\nu}_{\omega}(\mathbf{r})$ denotes the cold plasma dielectric tensor, similarly we denote its parameters. 
We added explicitely dependence on space in parameters that depend on the spatial variables. 

\item The case $\omega=\omega_c$:
Indeed, $\epsilon_{\infty}<\infty$ if $\omega_c^2\neq \omega^2$, but formally one also needs to require the boundedness of $\omega_p^2$. Therefore we 
added an assumption in the end of the Section Frequency Domain Study on the coefficients of the dielectric tensor, and made a remark on the absence
of the cyclotron resonance. The formulation of Lemma is correspondingly corrected. 

\item The limiting amplitude/absorption principle: 

A brief description of what limiting absorption and limiting amplitude principles had been added in the end of the introduction.
\end{itemize}


2. 'The reduction to a 1d problem is not clearly justified':  indeed, we changed this in the introduction by adding an assumption on $N_e, B_0$, 
and changed the wording in the introduction to underline that the 1D case is the one of interest. 


B. Specific remarks: 
\begin{itemize}
\item p1, abstract: changed

\item p2: in the frequency domain  the problem is regularized by a choice $|\nu|>0$, independently of the sign of $\nu$. 
It is interesting however that for a fixed $\omega$ the limits $\lim_{\nu\rightarrow 0\pm}$ do not result in the same solution. 
In the time domain the sign of $\nu>0$ is indeed of crucial importance for the stability, as well as in Lemma 2.1, provided that $\lambda$ in the 
frequency of the antenna is larger than zero. We added a remark on this before Section 2.1  

\item p2, after (1): $L$ is just a non-negative real number. We found it more convenient to place the point of the isolated hybrid resonance in $x=0$.
For $\nu\neq 0$ the energy decays, we added this in the introduction.


\item p3: this should be now fixed in the whole of the article.

\item p3 eq (4): the notation had been changed
  
\item p4, top: we changed the notation  

\item p4, eq. (5) fixed

\item p4, eq. (8): changed the notation in eq (8), though a more general result in Lemma 2.1 is formulated with $u,v$. 

\item p4: we changed $\alpha_{\nu}$ to $\beta_{\nu}$ not to confuse with $\alpha$ in the dielectric tensor. The index $\nu$ 
just underlines the dependence of the angle $\beta_{\nu}$ on $\nu$ (indeed, the coercivity result as it is stated, holds for this specific value of $\beta_{\nu}$ defined 
in the end of the proof of the lemma).

\item p. 5: indeed, we changed this 

\item p.6:  changed

\item p 6: $Ex$ changed

\item p 8: 'our' code, changed

 \item p.8: fixed section number
 
 \item p.8 eq (12): added $\nu=0$ + explanation
 
 \item p.8: added the Airy equation and pointed out the analyticity of the Airy equation. Additionally added a remark on the behaviour of the Airy function to justify the choice of the right boundary 
 condition $\partial_x E=0$. 
 
 \item p.8: the citation had been fixed; we expect $E_y$ to be analytic in this case (since we want it to be equal $\mathrm{Ai}(x)$, this is pointed out right now), 
 and $E_x$ thus is smooth.
 
 \item  p9, Figure 3: fixed the captions. This had been done for other figures as well. The values $E^{c}_{x,y}$ had been defined in the caption.
 
 \item p9 on top
  The title of Section 4.1.2 had been changed to Solution of a Frequency-Domain Problem with Resonance
  
 \item p9 Equation (15,16): changed  

 \item p9 after (17): it is really necessary to assume that $H^2$-norm that is bounded, for using the inequality (25). 
 We fixed an assumption on $E_y$, which was incorrect before.


  
 \item p10 before and after Table 1 : fixed
 
 \item p 11 Figure 4: fixed
 
 \item p11 Figure 5:
Agree, fixed. We added a remark in a caption to Figure 4 as well.


\item {\textcolor{red}{p12}}


\item p12 Section 4.3: this explanation had been moved to the introduction



\item p12 Section 4.3.1

  True, but in this case there is no resonance, since $\delta(x)\equiv 0$. Hence $E_x=\frac{i\delta}{\alpha}E_y\equiv 0$, 
  and $E_y$ solves the Airy equation in the frequency domain. We added a small remark to Section 4.3.1 to underline this. 

  
\item p12 Equation (23): it had been removed
  
\item p14 Figures 8 and 9

Added the captions  


\item p15 in Conclusion

  Indeed $P1$
  
\item references: corrected
  \end{itemize}


\section{Response to  Reviewer 2}
 \begin{itemize}
\item p2:  We tried to change the wording in the introduction to make it more clear what is a hybrid resonance. 
The singularity appears for $\nu=0$, provided that the diagonal part of the dielectric tensor vanishes and the non-diagonal part does not. 

\item 'matricial' had been changed to 'matrix'.

\item p6, 3.1. 

We employed the Yee scheme to discretize the problem in the time domain, and the finite element scheme solely for the frequency domain, 
and considered them independently. We added a remark on that in the very beginning of Section 3. 

\item p10: mesh multiple times finer. It should be said exactly how many times. 

We added a remark that it was at least two times for very fine meshes, and more for coarser ones.

\item {\textcolor{red}{p.12: fixed the typos.}}
\end{itemize}

  
 



\end{document}
