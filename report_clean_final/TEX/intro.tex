\section{Introduction}
Modeling various phenomena in plasmas is of practical importance for developing new sources of energy 
based on plasma fusion, see the ITER project\footnote{www.iter.org}. 
This article concentrates on studying a phenomenon of hybrid resonance \cite{Stix}, 
which is observed in experiments (see \cite{reflectometers_2006, reflectometers_2010, Dumont_2005}) and is described
mathematically as the non-regularity of
the solutions of Maxwell equations in magnetized plasmas \cite{Despres_2014}. 
Physically the hybrid resonance corresponds to situations where the electric field becomes
infinite due to some resonance coupling between the background magnetic field and
the polarization of the wave.
The energy deposit is resonant and may exceed by far the energy 
exchange which occurs in Landau damping, see \cite{Freidberg_2007}. 
Contrary to the Landau damping, however, 
hybrid resonance appears in a simpler one-dimensional  model coupling 
fluid equations with the non-electrostatic part of Maxwell equations.
Since  the mathematics of hybrid resonance is far to be understood
in dimension greater or equal to two, we restrict this work to dimension one.

To illustrate the hybrid resonance, we consider in this work a %one dimensional 
 time-dependent wave propagation problem leading to the cold plasma model for a single species (\nrev{electrons}) 
under a uniform in time background magnetic field $\mathbf{B}_0=(0,\;0,\; B_0)$. 
We restrict our attention to the poloidal plane, which is the plane 
perpendicular to the direction of the magnetic field. The model can be described by the time-dependent Maxwell system  \cite{Stix}
\begin{align}
\label{eq:main_system}
\begin{split}
-\epsilon_0 \partial_t E_x +\partial_y H_z = J_x,\\
-\epsilon_0 \partial_t E_y -\partial_x H_z = J_y,\\
\mu_0 \partial_t H_z + \partial_x E_y-\partial_y E_x= 0,
\end{split}
\end{align}
coupled through the linear electronic current  $\J = eN_e(\mathbf{r}) \ubf$
 with the equations 
 \begin{align}
\begin{split}
\label{eq:electronmove}
m_e \partial_t u_x =\epsilon_0 e E_x+e B_0(\mathbf{r}) u_y -m_e \nu u_x,\\
m_e \partial_t u_y =\epsilon_0 e E_y-e B_0(\mathbf{r}) u_x -m_e \nu u_y,
\end{split}
\end{align}
where $\nu\geq 0$ is the friction  between particles, $e<0$ is the charge of electrons, $m_e$ the electron mass and $N_e$ the electron density. In the cold plasma model, because of the frequencies considered, the electron density 
depends on space and is uniform in time. A non-zero value of $\nu$ constitutes a basis for the limiting absorption principle. 
The unknowns are the electron velocity $\ubf$ and the electromagnetic fields. 
The energy of this system in a domain $\Omega\in\mathbb R^2$ can be expressed as, see \cite{Despres_2014},
\ben
{\mathcal E}(t)= \int_\Omega \left(
\frac{\epsilon_0 |\E(t,\mathbf{r})|^2}{2}+\frac{ |\B(t,\mathbf{r})|^2}{2\mu_0}+\frac{m_e|\J (t,\mathbf{r})|^2}{2|e|N_e(\mathbf{r})} 
\right)\mathrm{d}\mathbf{r},
\een
and (\ref{eq:main_system}-\ref{eq:electronmove}) yield
\ben
\frac{d}{dt}{\mathcal E}=-\nu\left(\|u_{x}\|_{L^{2}(\Omega)}^2+\|u_{y}\|_{L^{2}(\Omega)}^2\right)+\text{boundary terms}.
\een
To illustrate unexpected difficulties that can occur in this system, consider the frequency domain counterpart of (\ref{eq:main_system}-\ref{eq:electronmove}), which corresponds to the cold plasma model. 
We introduce the plasma frequency $\omega_p(\mathbf{r})=\sqrt{\frac{|e|^2N_e(\mathbf{r})}{m\epsilon_0}}$, 
the cyclotron frequency $\omega_c(\mathbf{r})=\frac{e B_0(\mathbf{r})}{m_e}$ and perform the Fourier transform 
in time of (\ref{eq:main_system}-\ref{eq:electronmove}) (with the convention $\partial_t\rightarrow -i\omega$), for a 
particular case $\nu=0$:
\begin{align}
\label{eq:model_freq_domain}
 \mathbf{curl}\  {\operatorname{curl}}\  \hat{\mathbf{E}}-\frac{\omega^2}{c^2}\uuline{\varepsilon_{\omega}^{0}}(\mathbf{r})\hat{\mathbf{E}}=0,
\end{align}
where the cold-plasma dielectric tensor \cite[Chapter 1-2]{Stix}
\begin{align}
\nonumber
\uuline{\varepsilon_{\omega}^{0}}(\mathbf{r})=\left(
 \begin{matrix}
  \alpha_{\omega}(\mathbf{r})& -i\delta_{\omega}(\mathbf{r})\\
  i\delta_{\omega}(\mathbf{r}) & \alpha_{\omega}(\mathbf{r})
 \end{matrix}
\right),\\
\label{eq:alpha_delta}
\alpha_{\omega}(\mathbf{r})=1-\frac{\omega_p^2(\mathbf{r})}{\omega^2-\omega_c^2(\mathbf{r})},\; \delta_{\omega}(\mathbf{r})=\frac{\omega_c(\mathbf{r})\omega_p^2(\mathbf{r})}{\omega(\omega^2-\omega_c^2(\mathbf{r}))}.
\end{align}
The point $\mathbf{r}_0$ s.t. $\omega_c(\mathbf{r}_0)=\omega$, i.e. when $\uuline{\varepsilon_{\omega}^{0}}(\mathbf{r}_0)$ is singular, is called a cyclotron resonance (c.f. \cite[Chapter 1-5]{Stix}). 
In \cite{singular_solutions} it was demonstrated for a 1D-counterpart of the above system that such points behave like removable singularities, 
hence this case is not of interest for the present work. 

 The hybrid resonance is precisely defined 
 where  $\omega_p^2(\mathbf{r})+\omega_c^2(\mathbf{r})=\omega^2$ and $\omega_c(\mathbf{r})\omega_p(\mathbf{r})\neq 0$, i.e. when the diagonal part of the tensor 
$\uuline{\varepsilon_{\omega}^{0}}(\mathbf{r})$ vanishes and the non-diagonal part remains non-zero.
As shown in \cite{Despres_2014, singular_solutions}, for a 1D-counterpart of (\ref{eq:model_freq_domain}), in this case the time-harmonic electric field component \nrev{$\hat{E}_x$} is not necessarily square 
integrable (for a demonstration of this behaviour with the help of a simpler example, namely the Budden problem, see e.g. \cite{Despres_2014}). 
This apparent paradox is of course the source of important numerical difficulties which are the subject of the present study. 


Before proceeding, let us define the geometrical setting of the problem. 
We consider the frequency domain problem (\ref{eq:model_freq_domain}) posed in a truncated infinite half-plane, $(-L,\;H)\times\mathbb{R}$, for some $L>0$ and $H>0$, with the following
 boundary conditions. At the left boundary of the domain, which represents the wall of the Tokamak, we impose a Robin boundary condition
\be
\label{eq:boundary_conditions_intro}
%curl
 {\operatorname{curl}} \
  \hat{\mathbf{E}} +i \lambda\hat{\mathbf{E}}\wedge \n = \hat{g}_{inc} \equiv
   {\operatorname{curl}} \ \E_{inc} + i  \lambda\E_{inc}\wedge \n,
\ee
where $\E_{inc} = \exp \left(i  \lambda x\right)\begin{pmatrix} E_1\\ E_2 \end{pmatrix}$ and $\lambda\geq 0$ is 
the frequency of the antenna. At the right boundary we impose $\operatorname{curl} \hat{\E} = 0$. 


In this work we focus on stratified medium case: we suppose that
%a simplified model (which nevertheless is of physical interest when modelling antennas in plasmas),  where it is required to study the above system under the assumption that 
the electron density $N_{e}$ and the magnetic field $B_0$ are \textbf{uniform} in the second coordinate $y$,
$N_{e}=N_{e}(x)$ and $B_0=B_0(x)$, see e.g. \cite{Despres_2014}. 
We additionally assume that $N_e>0$ on $[-L,\;H]$. We choose this simplified case to focus on the fundamental mathematical difficulties of the hybrid resonance \cite{Despres_2014, singular_solutions}. 

Under the aforementioned assumptions on $N_e$ and $B_0$, the one-dimensional model can be derived from (\ref{eq:main_system}) by performing the Fourier transform in $y$ ($\partial_y\rightarrow i\theta, \; \theta\in\mathbb{R}$):
\begin{equation}
 \label{eq:main_model}
\begin{cases}
\epsilon_0\partial_t E_{x}-i\theta H_z=-eN_e u_x,\\
\epsilon_0\partial_t E_{y}+\partial_x H_z=-eN_e u_y,\\
\mu_0\partial_t H_z+\partial_x E_y-i\theta E_x =0,\\
m_e\partial_t u_x=eE_x+eu_yB_0-\nu m_e u_x,\\
m_e\partial_t u_y=eE_y-eu_xB_0-\nu m_e u_y,
\end{cases}
\end{equation}
 posed in the one-dimensional domain $(-L,\; H)$, with the corresponding boundary conditions:
 \begin{align}
 \label{eq:boundary_conditions_frequency_domain}
 \begin{split}
  \left.\left(\partial_x E_y -i\theta E_x+i\lambda E_y\right)\right|_{x=-L} =g_{inc}(t),\\
  \left.\partial_xE_y\right|_{x=H}=0, 
  \end{split}
 \end{align}
and the initial conditions for values $\left.E_{x}\right|_{t=0},\; \left.u_{x}\right|_{t=0},\; \left.E_{y}\right|_{t=0}, \; \left.u_{y}\right|_{t=0}, \; \left.H_z\right|_{t=0}$. 
This is the time-domain model that will be considered in the rest of the article, unless stated otherwise. 

The frequency domain formulation of (\ref{eq:main_model}) for $\nu=0$ is
\begin{align}
\label{eq:main_frequency_domain_intro}
\left(
\begin{matrix}
 i\theta \partial_x \hat{E}_y+\theta^2 \hat{E}_x\\
 i\theta \partial_x \hat{E}_x -\partial_x^2 \hat{E}_y
\end{matrix}
\right)-\frac{\omega^2}{c^2}
\uuline{\varepsilon_{\omega}^{0}}(x)\left(
\begin{matrix}
 \hat{E}_x\\
 \hat{E}_y
\end{matrix}
\right)
=0,
\end{align}
see (\ref{eq:alpha_delta}). Indeed, in a stratified medium, the coefficients $\alpha_{\omega},\;\delta_{\omega}$ depend solely on $x$ and $\omega$. We concentrate on a physical situation of the hybrid resonance, which we define as in \cite{singular_solutions}.
\begin{df}[of a hybrid resonance]
Let $\omega\in\mathbb{R}$ be fixed.
We will call a point $x_h\in \mathbb{R}$ a hybrid resonance if $\alpha_{\omega}$ has
a simple root in $x_h$ and has no other roots in an $\epsilon$-vicinity of this point $B_{\epsilon}(x_h)$, for some $\epsilon>0$,
and $\delta_{\omega}(x_h)\neq 0$.
The parameter $\nu$ is at the same the physical friction and a regularization parameter. The hybrid resonance singular solution
shows up at $x_h$ at the limit $\nu\rightarrow 0^+$.
\end{df}
To investigate the behaviour of the solutions at the hybrid resonance, 
we will study a sequence of the regularized problems, for $\nu\rightarrow 0$. These problems are not the frequency domain expression of \eqref{eq:main_model}, but a simple regularization of \eqref{eq:main_frequency_domain_intro}. This choice will be justified in further sections. The equations read: 
\begin{align}
\label{eq:seq_regularized}
\left(
\begin{matrix}
 i\theta \partial_x \hat{E}_y+\theta^2 \hat{E}_x\\
 i\theta \partial_x \hat{E}_x -\partial_x^2 \hat{E}_y
\end{matrix}
\right)-\frac{\omega^2}{c^2}
\left(\uuline{\varepsilon_{\omega}^{0}}(x)+i\nu \mathrm{Id}\right)\left(
\begin{matrix}
 \hat{E}_x\\
 \hat{E}_y
\end{matrix}
\right)
=0,
\end{align}
where $\mathrm{Id}$ is the identity matrix. 

The goal of this article is three-fold. 
First, we investigate the finite element approximation of the 1D 
frequency domain problem (\ref{eq:seq_regularized}), with a regularized dielectric tensor for a small parameter $\nu$. We prove the well-posedness of this problem for $\nu\neq 0$ in Section~\ref{sec:wellposedness} and 
demonstrate that the use of  $P_1$ finite elements allows to approximate the singularity 
of the solution efficiently (Section \ref{sec:freq_dep}). Second, we develop an original scheme based on widely appreciated semi-lagrangian schemes
for the discretization of time domain Maxwell's equations with a linear current.
Third, we consider the case $\nu\rightarrow 0$, and study the limiting amplitude solution 
$\lim\limits_{t\rightarrow +\infty}\lim\limits_{\nu\rightarrow 0}\mathbf{E}(t)$ obtained with the help of 
the FDTD discretization of \eqref{eq:main_model}, suggested in \cite{stable_yee_plasma_current}. 
We compare this result with 
$\hat{\mathbf{E}}\mathrm{e}^{-i\omega t}$, computed in the frequency domain, for $\nu\rightarrow 0$, which corresponds to
$\lim\limits_{\nu\rightarrow 0}\lim\limits_{t\rightarrow+\infty}\mathbf{E}(t)$.
Such a comparison is a way to study the formal commutation relation between the limited absorption and the limiting amplitude principles. 
In application to a non-resonant case of (\ref{eq:main_frequency_domain_intro}), equipped with boundary conditions, the limited absorption principle states that the solution of (\ref{eq:seq_regularized}) 
would approach the solution of (\ref{eq:main_frequency_domain_intro})
as $\nu\rightarrow 0$. Similarly, the limiting amplitude principle applied to the problem (\ref{eq:main_model}) 
states that if the boundary conditions are chosen harmonic in time, e.g. $\left.\partial_x E_y\right|_{x=-L}=\mathrm{e}^{-i\omega_{*} t}$, then the solution of this problem $(\mathbf{E}(t),\;H(t))$ asymptotically ($t\rightarrow +\infty$)
tends to a steady state of the form $ \left(\hat{\mathbf{E}}(\omega_{*}),\; \hat{H}(\omega_{*})\right)\mathrm{e}^{-i\omega_{*} t}$, where 
$\left(\hat{\mathbf{E}}(\omega_{*}),\; \hat{H}(\omega_{*})\right)$ is the solution of (\ref{eq:main_model})
in the frequency domain, with $\omega=\omega_{*}$ and the boundary condition $\left.\partial_x \hat{E}_y\right|_{x=-L}=1$. For more details see e.g. \cite{Morawetz, Eidus}. The commutation between the two principles then writes as
$
\lim\limits_{\nu\rightarrow 0}\lim\limits_{t\rightarrow+\infty}= \lim\limits_{t\rightarrow+\infty}\lim\limits_{\nu\rightarrow 0}$,
see Figure \ref{fig:limits}. 
\begin{figure}
\begin{tikzpicture}
 \matrix (m) [matrix of math nodes,row sep=3em,column sep=4em,minimum width=2em] {
\mathbf{E}^{\nu}(t) & \mathbf{E}^{0+}(t)\\
\hat{\mathbf{E}}^{\nu}(t)\mathrm{e}^{i\omega t} &
\lim\limits_{\nu\rightarrow 0}\lim\limits_{t\rightarrow \infty}\mathbf{E}^{\nu}(t)=\lim\limits_{t\rightarrow \infty}\lim\limits_{\nu\rightarrow 0}\mathbf{E}^{\nu}(t)\\
};
\path[-stealth,font=\scriptsize] (m-1-1) edge node [left] {$t\rightarrow+\infty$} (m-2-1);
\path[-stealth, font=\scriptsize] (m-1-1) edge node [above] {$\nu\rightarrow 0+$} (m-1-2);
\path[-stealth, font=\scriptsize] (m-2-1) edge node [above] {$\nu\rightarrow 0$} (m-2-2);
\path[-stealth, font=\scriptsize] (m-1-2) edge node [right] {$t\rightarrow +\infty$} (m-2-2);
% \path[-stealth] (m-2-1.east) edge node [above] {$\nu\rightarrow 0$} (m-2-2);
%(m-1-2) edge node [right] {$\nu\rightarrow 0$} (m-2-2);
\end{tikzpicture}
 \caption{Schematic representation of the equivalence of the limited absorption and limiting amplitude principles.}
\label{fig:limits}
\end{figure}
Even though it is true for standard linear wave problems, it is not clear if it still holds in the singular case 
of the hybrid resonance.


To our knowledge, such numerical studies have not been performed in the existing literature. 