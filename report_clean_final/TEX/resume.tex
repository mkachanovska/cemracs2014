La r\'esonnance hybride est un ph\'enom\`ene physique qui apparait par exemple lorsque l'on chauffe un plasma.
Il est en particulier d'int\'er\^et scientifique dans le cadre du d\'eveloppement  du projet ITER. Dans ce papier, nous nous concentrons sur des solutions faiblement r\'eguli\`eres des \'equations de Maxwell pour les plasmas magn\'etiques. Notre but est ici triple. D'un c\^ot\'e nous 
approchons num\'eriquement la formulation fr\'equentielle \`a l'aide d'\'el\'ements finis,
 et de l'autre nous \'etudions les solutions r\'esonnantes dans le domaine temporel, \`a l'aide de deux m\'ethodes distinctes de diff\'erences finies. Finalement nous comparons  num\'eriquement les solutions fr\'equentielles avec le comportement en temps long des solutions temporelles, dans le cadre
des principes d'amplitude et d'absorption limite.

